\section{Бесконечные произведения}
\begin{definition}
  Пусть $\{b_n\}$ --- поселдовательность положительных чисел. Формально
  бесконечное произведение ее членов $b_1b_2\dots b_n\dots$ называется
  бесконечным произведением. Обозначается так:
  \begin{gather*}
    b_1b_2\dots = \prod\limits_{n = 1}^\infty b_n = \prod b_n
  \end{gather*}
\end{definition}

\begin{definition}
  Конечное произведение ${\prod}_n = b_1\dots b_n$ называется $n$-ым
  частичным произведением.
\end{definition}

\begin{definition}
  Если последовательность произведений ${\prod}_n$ сходится к числу $\prod \not =
  0 \ (\prod > 0)$, то бесконечное произведение называется сходящимся к
  $\prod$. Если $\prod = 0$, то бесконечное произведение расходится к нулю, а
  если $\prod_n \to \infty$, то оно называется расходящимся к бесконечности.
  Если предела нет, то оно называется расходящимся.
\end{definition}

\begin{approval}[необходимый признак сходимости]
  Если $\prod b_n$ сходится, то $b_n \to 1$ при $n \to \infty$.
\end{approval}

\begin{proof}
  Если $\prod_n \to \prod \not = 0$, то
  \begin{gather*}
    b_n = \frac{{\prod}_n}{{\prod}_{n-1}} \to
    \frac{\prod}{\prod} = 1, \ n \to \infty
  \end{gather*}
\end{proof}

\begin{approval}
  Сходимость бесконечного произведения $\prod b_n$ влечет за собой сходимость
  ряда $\sum \ln b_n$ и наоборот. Причем
  \begin{gather*}
    \ln \prod\limits_{n = 1}^\infty b_n = \sum\limits_{n = 1}^{\infty} \ln b_n
  \end{gather*}
\end{approval}

\begin{proof}
  Имеем
  \begin{gather}
    \ln {\prod}_n = \sum\limits_{k = 1}^{n} \ln b_k =: S_n, \ {\prod}_n = e^{S_n}
    \label{app912:eq1}
  \end{gather}
  из непрерывности логарифмической и показательной функции следует, что если
  ${\prod}_n \to \prod$, то $S_n \to \ln \prod$ и обратно, если $S_n \to S$, то
  ${\prod}_n \to e^{S}$.
\end{proof}

\begin{definition}
  Бесконечное произведение называется абсолютно сходящимся, если абсолютно
  сходится ряд $\sum \ln b_n$. Сходящееся произведение, не являющееся абсолютно
  сходящимся, называется условно сходящимся. Очевидно, что абсолютно сходящееся
  произведение сходится в обычном смысле. Поскольку мы считаем $b_n > 0$, то
  $b_n = 1 + a_n$, где $a_n > -1$. Тогда имеем:
  \begin{gather*}
    \prod\limits_{n = 1}^\infty b_n = \prod\limits_{n = 1}^\infty (1 + a_n)
  \end{gather*}
\end{definition}

\begin{theorem}
  \label{th911}
  Бесконечное произведение $\prod\limits_{n = 1}^\infty (1 + a_n)$ абсолютно
  сходится тогда и только тогда, когда $\sum\limits_{n = 1}^{\infty} |a_n|$
  сходится.
\end{theorem}

\begin{proof}
  Действительно, $1+a_n \to 1 \Longleftrightarrow a_n \to 0$, так что $|\ln (1 +
  a_n)| \sim |a_n|, n \to \infty$. Ряды $\sum |\ln (1 + a_n)|$ и $\sum |a_n|$
  сходятся и расходятся одновременно.
\end{proof}

\section{Определение $\Gamma$-функции Эйлера. Некоторые ее свойства}
По определению $\Gamma(x)$
\begin{gather}
  \Gamma(x) = \frac{1}{x} \prod\limits_{n = 1}^{\infty} \frac{(1 +
  \frac{1}{n})^x}{1 + \frac{x}{n}}, \ x \not = 0, -1, -2, \dots
  \label{ch92:eq1}
\end{gather}
Представим общий член произведения так:
\begin{gather*}
  \frac{(1 + \frac{1}{n})^x}{1 + \frac{x}{n}} = 1 + \frac{x(x - 1)}{2} \cdot
  \frac{1}{n^2} + \bar{\bar{o}}\left(\frac{1}{n^2}\right), \ n \to \infty
\end{gather*}
Отсюда в силу теоремы \eqref{th911} вытекает, что наше параметрическое
представление абсолютно сходится. \\

Отметим некоторые свойства $\Gamma$-функции Эйлера:
\begin{enumerate}
  \item
    \begin{gather}
      \Gamma(x) = \lim\limits_{n \to \infty} P_n(x), \ P_n(x) = (n+1)^x
      \frac{n!}{x(x+1)\dots(x+n)}
      \label{ch92:eq2}
    \end{gather}
    Из того, что $n$-ое частичное произведение имеет вид:
    \begin{gather*}
      \frac{\prod\limits_{k = 1}^n \left(\frac{k + 1}{k}\right)^x}{x(1 + x)(1 +
      \frac{x}{2})\dots(1+\frac{x}{n})} = \frac{(n + 1)^x n!}{x(x + 1) \dots (x
      +n)} = P_n(x)
    \end{gather*}
  \item Функциональное уравнения для $\Gamma$-функции:
    \begin{gather}
      \Gamma(x + 1) = x \Gamma(x), \ \Gamma(1) = 1
      \label{ch92:eq3}
    \end{gather}
    действительно, $\frac{\Gamma(x + 1)}{\Gamma(x)} = \lim\limits_{n \to
    \infty} \frac{P_n(x + 1)}{P_n(x)} = \lim\limits_{n \to \infty} \frac{(n +
    1)x}{(x + 1) + n} = x$. \\
    В качестве следствия получаем: $\Gamma(m + 1) = m!$.
\end{enumerate}

\section{Представления синуса в виде бесконечного произведения и формула
дополнения для $\Gamma$-функции}
\begin{theorem}[разложение функции $\ctg$ на простейшие дроби]
  Имеет место формула:
  \begin{gather}
    \pi \ctg \alpha \pi = \frac{1}{\alpha} + \sum\limits_{k = 1}^{\infty}
    \frac{2\alpha}{\alpha^2 - k^2} = \lim\limits_{n \to \infty} \sum\limits_{k
    = -n}^{n} \frac{1}{\alpha - k}, \ \alpha \in \mathbb{R} \ (\mathbb{Z})
    \label{th931:eq1}
  \end{gather}
\end{theorem}
\begin{proof}
  Формулу \eqref{th931:eq1} достаточно установить для $\alpha \in (0,
  \frac{1}{2})$. Справедливость ее для остальных $\alpha$, отличных от
  $\mathbb{Z}$ следует из нечетности выражений слева и справа в формуле
  \eqref{th931:eq1} и их периодичности. \\ На отрезке $[-\pi, \pi]$ рассмотрим
  функцию $g(x) = \cos \alpha x$, где $\alpha \in [-\frac{1}{2}, \frac{1}{2}]$.
  Продолжим ее на $\mathbb{R}$. Она будет периодической с периодом $2\pi$.
  Тогда $g(x)$ будет четной и непрерывной на $\mathbb{R}$. Так как $g(x)$ ---
  непрерывная и кусочно-гладкая функция, то ее ряд Фурье равномерно сходится на
  $I = [0, 2\pi]$ к функции $g(x)$. Поэтому при всех $x \in I$:
  \begin{gather*}
    g(x) = \frac{a_0}{2} + \sum\limits_{k = 1}^{\infty} (a_k \cos kx + b_k
    \sin kx)
  \end{gather*}
  где $b_k = 0$ в силу четности $g(x)$.
  \begin{gather*}
    \frac{a_0}{2} = \frac{\sin \alpha \pi}{\alpha \pi} \\
    a_k = (-1)^k \frac{2\alpha}{\alpha^2 - k^2} \cdot \frac{\sin \alpha
    \pi}{\pi}
  \end{gather*}
  таким образом:
  \begin{gather*}
    g(x) = \cos \alpha x = \frac{\sin \alpha \pi}{\pi} \left(\frac{1}{\alpha} +
    \sum\limits_{k = 1}^{\infty} (-1)^k \frac{2\alpha}{\alpha^2 - k^2} \cos kx
    \right), \ x = \pi
  \end{gather*}
\end{proof}

\begin{theorem}[Эйлера]
  $\forall \alpha \in \mathbb{R}$,
  \begin{gather}
    \sin \pi \alpha = \pi \alpha \prod\limits_{k = 1}^\infty \left(1 -
    \frac{\alpha^2}{k^2}\right)
    \label{th932:eq1}
  \end{gather}
\end{theorem}

\begin{proof}
  Равенство \eqref{th932:eq1} эквивалентно при $\alpha \in (0, 1)$ равенству:
  \begin{gather}
    \ln \left(\frac{\sin \pi \alpha}{\pi \alpha}\right) = \sum\limits_{k =
    1}^{\infty} \ln \left(1 - \frac{\alpha^2}{k^2} \right) =: S(\alpha)
    \label{th932:eq2}
  \end{gather}
  Ряд справа сходится равномерно при $\alpha \in [-\alpha_0, \alpha_0]$, где $0
  < \alpha_0 < 1$. \\
  Пусть $S(\alpha)$ --- сумма этого ряда. Так как:
  \begin{gather*}
    \ln \left(1 - \frac{\alpha^2}{k^2}\right) = \frac{2\alpha}{\alpha^2 - k^2}
  \end{gather*}
  то $\sum\limits_{k = 1}^{\infty} \frac{2\alpha}{\alpha^2 - k^2}$
  равномерно сходится на $[-\alpha_0, \alpha_0]$, то существуют произведения
  $S'(\alpha)$ и выполняется равенство:
  \begin{gather*}
    S'(\alpha) = \sum\limits_{k = 1}^{\infty} \frac{2\alpha}{\alpha^2 - k^2}, \
    \alpha \in (-1, 1)
  \end{gather*}
  отсюда и из \eqref{th931:eq1} получим:
  \begin{gather*}
    S'(\alpha) = \pi \ctg \alpha \pi - \frac{1}{\alpha}, \ \alpha \in (-1, 1) \\
    S(\alpha) = \ln \frac{\sin \pi \alpha}{\alpha} + C, \ C = const
  \end{gather*}
  так как $\lim\limits_{\alpha \to 0} S(\alpha) = 0$ ряд \eqref{th932:eq2}
  сходится равномерно, то
  \begin{gather*}
    C = - \lim \ln \frac{\sin \pi \alpha}{\alpha} = - \ln \pi
  \end{gather*}
  так что
  \begin{gather*}
    S(\alpha) = \ln \frac{\sin \pi \alpha}{\pi \alpha}
  \end{gather*}
  и равенство \eqref{th932:eq2}, а также \eqref{th932:eq1} доказаны.
\end{proof}

\begin{consequence}
  При $\alpha = \frac{1}{2}$ получим формулу Валлиса:
  \begin{gather}
    \frac{2}{\pi} = \prod\limits_{k = 1}^\infty \left(1 - \frac{1}{4n^2}\right)
    \label{th932:eq3}
  \end{gather}
\end{consequence}

\begin{theorem}[формула дополнения Эйлера]
  При $x \not \in \mathbb{Z}$:
  \begin{gather}
    \Gamma(x) \Gamma(1 - x) = \frac{\pi}{\sin \pi x}
    \label{th933:eq1}
  \end{gather}
\end{theorem}

\begin{proof}
  Используя определение $\Gamma$-функции и следующие равенство:
  \begin{gather*}
    \Gamma(1 - x) = -x \Gamma(-x) = \prod\limits_{n = 1}^\infty
    \frac{\left(1 + \frac{1}{n}\right)^{-x}}{1 - \frac{x}{n}}
  \end{gather*}
  докажем эту теорему. Перемножая $\Gamma(x)$ и $\Gamma(1 - x)$, получим:
  \begin{gather*}
    \Gamma(x) \cdot \Gamma(1 - x) = \frac{\pi}{\sin \pi x}
  \end{gather*}
\end{proof}

\begin{consequence}
  При $x = \frac{1}{2}$ находим:
  \begin{gather}
    \Gamma \left( \frac{1}{2} \right) = \sqrt{\pi}
  \end{gather}
\end{consequence}

\section{Интегральное представление для $\Gamma$-функции Эйлера}

\begin{lemma}[формула Гаусса]
  Пусть $P_n(x) = (n + 1)^x \frac{n!}{x(x + 1)\dots(x + n)}$. При $x > 0$:
  \begin{gather*}
    P_n(x) = \left( 1 + \frac{1}{n} \right)^x \int\limits_0^n \left( 1 -
    \frac{t}{n} \right)^n t^{x - 1} dt
  \end{gather*}
\end{lemma}

\begin{proof}
  С помощью замены переменной и интегрирования по частям, получаем:
  \begin{gather*}
    \left( 1 + \frac{1}{n} \right)^x \int\limits_0^n \left( 1 -
    \frac{t}{n} \right)^n t^{x - 1} dt = (n + 1)^x \int\limits_0^1 (1 - y)^n
    y^{x - 1} dy = (n + 1)^x \frac{1}{x} \int\limits_0^1 (1 - y)^n dy^x = \\
    (n + 1)^x \frac{n}{x} \int\limits_0^1 (1 - y)^{n - 1}y^x dy = \dots = \\
    (n+1)^x \frac{n!}{x(x + 1)\dots(x + n -1)} \int\limits_0^1 y^{x + n - 1} dy
    = (n + 1)^x \frac{n!}{x(x + 1) \dots (x + n)} = P_n(x)
  \end{gather*}
\end{proof}

\begin{theorem}[Интегральное представление $\Gamma$-функции]
  \label{th941}
  $x > 0$
  \begin{gather}
    \Gamma(x) = \int\limits_0^\infty t^{x-1} e^{-t} dt
    \label{th941:eq1}
  \end{gather}
  При $0 < x < 1$: Интеграл имеет особую точку $t = 0$ и сходится в ее
  окрестности. В окрестности $\infty$ подинтегральное выражение ограничено
  функцией $e^{-t/2}$ и сходится в этой окрестности.
\end{theorem}

\begin{proof}
  Рассмотрим разность
  \begin{gather}
    R_n(x) = \int\limits_0^n t^{x-1} e^{-t} dt - P_n(x) \left(1 +
    \frac{1}{n}\right)^{-x} = \int\limits_0^n t^{x - 1} \left( e^{-t} - \left(
    1 - \frac{t}{n} \right)^n \right) dt
    \label{th941:eq2}
  \end{gather}
  покажем, что при $t \in [0, n]$:
  \begin{gather}
    0 \leq e^{-t} - \left( 1 - \frac{t}{n} \right)^n \leq \frac{1}{n} e^{-t}
    t^2
    \label{th941:eq3}
  \end{gather}
  действительно, любое неравенство является следствием неравенства $1 + y \leq
  e^y\left( 1 - \frac{t}{n} \right)~\leq~e^{-t/n}$. Правое неравенство получено
  ссылкой на этоже неравенство, а так же на неравенство Бернулли:
  $(1 + y)^n \geq 1 + ny, \ (y > -1)$. Учитывая все это, получаем:
  \begin{gather*}
    e^{-t} - \left( 1 - \frac{t}{n} \right)^n = e^{-t}\left( 1 - e^t\left( 1 -
    \frac{t}{n} \right)^n \right) \leq e^{-t} \left( 1 - \left( 1 + \frac{t}{n}
    \right)^n \left( 1 - \frac{t}{n} \right)^n \right) = \\
    e^{-t} \left( 1 - \left( 1 - \frac{t^2}{n^2} \right)^n \right) \leq e^{-t}
    n\frac{t^2}{n^2} = \frac{1}{n} e^{-t} t^2
  \end{gather*}
  учитывая \eqref{th941:eq2} и \eqref{th941:eq3} получаем:
  \begin{gather*}
    0 \leq R_n(x) < \int\limits_0^n \frac{t^{x-1}e^{-t}}{n} dt <
    \frac{1}{n}\int\limits_0^\infty t^{x-1} e^{-t} dt = \frac{\Gamma(x)}{n}
    \Rightarrow R_n(x) \to 0, n \to \infty
  \end{gather*}
  отсюда $\int\limits_0^\infty t^{x-1} e^{-t} dt = \lim\limits_{n \to \infty}
  \int\limits_0^n t^{x-1} e^{-t} dt = \lim\limits_{n \to \infty} \left( R_n(x)
  + P_n(x) \left( 1 + \frac{1}{n} \right)^{-x} \right) = \lim\limits_{n \to
  \infty} P_n(x) = \Gamma(x)$.
\end{proof}

\begin{remark}
  С помощью интегрирования по частям и теоремы \eqref{th941} вводится следующая
  формула Коши, справедливая при значениях $x \in (-(n + 1), n), \ n \in
  \mathbb{N}$:
  \begin{gather*}
    \Gamma(x) = \int\limits_0^\infty t^{x-1} (e^{-t} - \varphi_n(t)) dt, \
    \varphi_n(t) = \sum\limits_{k = 0}^{n} (-t)^k \frac{1}{k!}
  \end{gather*}
\end{remark}

\begin{consequence}[интеграл Эйлера-Пуассона]
  \begin{gather*}
    \int\limits_0^\infty e^{-x^2} dx = \frac{\sqrt{\pi}}{2}, \
    \Gamma\left(  \frac{1}{2} \right) = \int\limits_0^\infty
    \frac{e^{-t}}{\sqrt{t}} dt = 2 \int\limits_0^{+\infty} e^{-x^2} dx, \
    \Gamma\left( \frac{1}{2} \right) = \sqrt{\pi}
  \end{gather*}
\end{consequence}

Дальнейшее изучение свойств $\Gamma$-функций будем проводить, исходя из ее
интегрального представления и основываясь на теории интегралов, зависящих от
параметров.

\section{Формула Стирлинга}

$\Gamma(x+1) = \sqrt{2\pi x} \left( \frac{x}{e} \right)^x\left( 1 +
\frac{1}{12x} + \frac{1}{288x^2} + \bar{\bar{o}}\left(\frac{1}{x^3}\right)
\right), \ x \to \infty$
\begin{theorem}
  \label{th951}
  \begin{gather}
    \Gamma(n + 1) = n! \sim \sqrt{2\pi n}\left( \frac{n}{e} \right)^n, \ n\to
    \infty, \ n \in \mathbb{N}
    \label{ch95:eq1}
  \end{gather}
  $(\sqrt[n]{n!} \sim \frac{n}{e}, \ n \to \infty)$.
\end{theorem}

\begin{lemma}
  \label{lem951}
  При $\forall n = 1, 2, \dots$ имеем:
  \begin{gather}
    e < \left( 1 + \frac{1}{n} \right)^{n + \frac{1}{2}} < e^{1 +
      \frac{1}{12n(n+1)}}
    \label{lem951:eq1}
  \end{gather}
\end{lemma}

\begin{proof}
  $|x| < 1, \ \ln \frac{1 + x}{1 - x} = \ln(1 + x) - \ln(1 - x) = \sum\limits_{n
  = 1}^{\infty} (-1)^{n+1} \frac{x^n}{n} - \sum\limits_{n = 1}^{\infty} \left(
  -\frac{x^n}{n} \right) = 2x \sum\limits_{k = 0}^{\infty} \frac{x^{2k}}{2k +
  1} = 2x(1 + \frac{1}{3}x^2 + \frac{1}{5}x^4 + \dots)$, отсюда при $0 < x < 1$
  получим:
  \begin{gather*}
    2x < \ln \frac{1 + x}{1 - x} < 2x\left( 1 + \frac{1}{3}\frac{x^2}{1 - x^2}
    \right) \\
    1 < \frac{1}{2x} \ln \frac{1 + x}{1 - x} < 1 + \frac{1}{3}\frac{x^2}{1 - x^2}
  \end{gather*}
  или, что тоже самое,
  \begin{gather*}
    1 < \frac{t}{2} \ln \frac{t + 1}{t - 1} < 1 + \frac{1}{3}\frac{1}{t^2-1}, \
    t > 1, \ t = 2n + 1, \ n \in \mathbb{N} \\
    1 < \left( n + \frac{1}{2} \right)\ln \left( 1 + \frac{1}{n} \right) < 1 +
    \frac{1}{12} \frac{1}{n(n+1)}
  \end{gather*}
  потенциируя, получаем \eqref{lem951:eq1}.
\end{proof}

\begin{lemma}[формула Валлиса]
  \label{lem952}
  \begin{gather}
    \pi = \lim\limits_{n \to \infty} \frac{1}{n} \left( 2^{2n}
    \frac{(n!)^2}{(2n)!} \right)^2
    \label{lem952:eq1}
  \end{gather}
\end{lemma}

\begin{proof}
  Действительно, согласно формуле \eqref{th932:eq3}:
  \begin{gather}
    \frac{\pi}{2} = \prod\limits_{k = 1}^\infty \left( 1 - \frac{1}{4k^2}
    \right)^{-1} = \lim\limits_{n \to \infty} \prod\limits_{k = 1}^n
    \frac{(2k)^2}{(2k - 1)(2k + 1)}
    \label{lem952:eq2}
  \end{gather}
  запишем ${\prod}_n$ в виде:
  \begin{gather*}
    {\prod}_n = \frac{1}{2n + 1} \left( \frac{(2n)!!}{(2n - 1)!!} \right)^2 =
    \frac{1}{2n + 1} \left( \frac{((2n)!!)^2}{(2n)!} \right)^2 = \frac{1}{2n +
    1}\left( \frac{2^{2n}(n!)^2}{(2n)!} \right)^2
  \end{gather*}
  отсюда и из \eqref{lem952:eq2} получаем \eqref{lem952:eq1}.
\end{proof}

\begin{proof}(теоремы \eqref{th951}).
  Положим:
  \begin{gather}
    x_n = \frac{n!e^n}{n^{n + \frac{1}{2}}}
    \label{th951:eq2}
  \end{gather}
  требуется доказать, что
  \begin{gather}
    \lim\limits_{n \to \infty} x_n = \sqrt{2\pi}
    \label{th951:eq3}
  \end{gather}
  поскольку $\frac{x_n}{x_{n+1}} = \frac{1}{e}\left( 1 + \frac{1}{n}
  \right)^{n+\frac{1}{2}}$, то в силу \eqref{lem951:eq1} получаем:
  \begin{gather}
    1 < \frac{x_n}{x_{n+1}} < \frac{e^{\frac{1}{12n}}}{e^{\frac{1}{12(n + 1)}}}
    = e^{\frac{1}{12n(n+1)}}
    \label{th951:eq4}
  \end{gather}
  из левого неравенства следует, что последовательность $x_n$ убывает и
  следовательно имеет конечный предел $a, \ (a \geq 0)$:
  \begin{gather}
    \lim\limits_{n \to \infty} x_n = a
    \label{th951:eq5}
  \end{gather}
  положим $y_n = x_n e^{-\frac{1}{12n}}, \ n \in \mathbb{N}$. Очевидно,
  $\lim\limits_{n \to \infty} y_n = a$. В силу правого из неравенств
  \eqref{th951:eq4} $y_n < y_{n+1}$, так что $a > 0$. Получаем, что $x_n = a(1
  + \varepsilon_n)$, где $\varepsilon_n \to 0, \ n \to \infty$. \\ Отсюда из
  \eqref{th951:eq2} получим:
  \begin{gather*}
    n! = a \frac{n^{n+\frac{1}{2}}}{e^n} \left( 1 + \varepsilon_n \right) \\
    (2n)! = a \frac{2^{2n + \frac{1}{2}} n^{2n + \frac{1}{2}}}{e^{2n}} (1 +
    \varepsilon_{2n})
  \end{gather*}
  Подставив эти выражения в формулу Валлиса \eqref{lem952:eq1} получаем:
  \begin{gather*}
    \pi = \lim\limits_{n \to \infty} \frac{1}{n} \left( a \sqrt{\frac{n}{2}}
    \cdot \frac{(1 + \varepsilon_n)^2}{1 + \varepsilon_{2n}} \right)^2 \\
    \pi = \lim\limits_{n \to \infty} \frac{1}{n} \left( 2^{2n}\left(
    \frac{a n^{n + \frac{1}{2}}(1+\varepsilon_n)}{e^n} \right)^2 \cdot
    \frac{1}{a} \cdot \frac{e^{2n}}{2^{2n + \frac{1}{2}} n^{2n + \frac{1}{2}}(1
    + \varepsilon_{2n})}\right)^2 = \\ \lim\limits_{n \to \infty}
    \frac{1}{n}\left( \frac{a(1 + \varepsilon_n)^2}{1 + \varepsilon_{2n}} \cdot
    \sqrt{\frac{n}{2}}\right)^2 = \frac{a^2}{2} \Longrightarrow a = \sqrt{2\pi}
  \end{gather*}
  что и требовалось доказать (смотри \eqref{th951:eq3} и \eqref{th951:eq5}).

\end{proof}
