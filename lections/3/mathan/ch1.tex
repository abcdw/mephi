\section{Определение}

\begin{definition}
  $U_1 + U_2 + U_3 + \dots = \sum\limits_{k = 0}^{\infty}{U_k}$
\end{definition}

\begin{definition}[Частичная сумма]
  $S_n = \sum\limits_{k = 0}^n U_k$
\end{definition}

\begin{definition}
  Ряд сходится, если $\exists \lim\limits_{n \to \infty}
                      \sum\limits_0^\infty U_k = S$
\end{definition}

\begin{definition}
  $\{a_n\} = a_0 + \sum\limits_1^n (a_k - a_{k-1})$, где $(a_k - a_{k - 1}) = U_k$
\end{definition}

\begin{theorem}[Критерий Коши]
  \label{th111}
  Ряд сходится, тогда и только тогда, когда он удовлетворяет условию Коши: \\
  $$\forall \varepsilon > 0, \ \exists N~=~N(\varepsilon), \
  \forall n \geq N, \forall p: \ |\sum\limits_{k=n+1}^{n+p} U_k|~=~|U_{n+1}
  + \dots + U_{n+p}|~=~|S_{n+p} - S_{n}| < \varepsilon$$
\end{theorem}

\begin{proof}
  $\sum U_k$ - сходится $\Leftrightarrow \{S_n\}$
  $$\forall \varepsilon > 0, \ \exists N: \forall n \geq N, \forall p: \
  |S_{n+p} - S_{n}| < \varepsilon$$
\end{proof}
\begin{consequence}[Необходимое условие сходимости]
  Если $\sum U_k$ сходится, то $U_k \to 0$, при $k~\to~\infty$
\end{consequence}
\begin{proof}
  Если $\sum U_k$ сходится, то выполняется Критерий Коши. При $p = 1$
  $$\forall \varepsilon > 0, \exists N, \forall n \geq N:
  |S_{n+1} - S_n|~=~|U_{n+1}| < \varepsilon$$
\end{proof}
\begin{consequence}
  Отбрасывание или добавление любого конечного числа членов ряда на его
  сходимость не влияет.
\end{consequence}

\begin{example}
  $\sum\limits_0^\infty z^n,$
  $S_n(z) = \sum\limits_0^n z_n = \frac{1-z^{n+1}}{1-z}$.
  При $n \to \infty, \ S_n(z) = \frac{1}{1 - z}, \ |z| < 1$. \\
  $S_n(z)$ не имеет придела при $|z| \geq 1$.
\end{example}

\section{Действия с рядами}

\begin{theorem}
  Ряды $\sum U_k$ и $\sum V_k$ сходятся,$\alpha$ --- комплексное число, тогда
  \begin{align}
    \label{series:property1}
    \sum\alpha U_k &= \alpha\sum U_k \\
    \label{series:property2}
    \sum (U_k \pm V_k) &= \sum U_k \pm \sum V_k
  \end{align}
\end{theorem}
\begin{proof}
  Доказательство свойства (\eqref{series:property1}):
  $$\sum\limits_{k=0}^\infty \alpha U_k = \lim\limits_{n \to \infty}
  \sum\limits_0^n \alpha U_k = \alpha \lim\limits_{n \to \infty} \sum\limits_0^n U_k
  = \alpha \sum\limits_{k = 0}^\infty U_k$$
\end{proof}
\begin{proof}
  Доказательство свойства (\eqref{series:property2}):
  \begin{gather*}
      \sum\limits_0^\infty U_k \pm \sum\limits_{0}^{\infty} V_k
      = \lim\limits_{n \to \infty} \sum\limits_{0}^{n} U_k
      \pm \lim\limits_{n \to \infty} \sum\limits_{0}^{n} V_k = \\
      = \lim\limits_{n \to \infty} \sum\limits_{0}^{n} U_k
      \pm \sum\limits_{0}^{n} V_k
      = \lim\limits_{n \to \infty} \sum\limits_{0}^{n} (U_k \pm V_k)
      = \sum\limits_{0}^{n} (U_k \pm V_k)
  \end{gather*}
\end{proof}

\begin{remark}
  Из сходимости $\sum (U_k \pm V_k) \not \Rightarrow$ сходимость
  $\sum U_k$ и $\sum V_k$
\end{remark}
\begin{remark}
  Если $\sum U_k$ сходится, то можно группировать, не меняя порядка.
\end{remark}
\begin{example}
  \begin{gather*}
    \sum (1 - 1) \\
    (1 - 1) + (1 - 1) + \dots \\
    1 - (1 - 1) - (1 - 1) \dots
  \end{gather*}
\end{example}
\begin{comment}
  Нельзя раскрывать скобки и переставлять члены.
\end{comment}
\subsection{Ряды с неотрицательными членами}
$U_k \geq 0$, $S_n = \sum\limits_{0}^{n} U_k$ - не убывающая последовательность. \\
$\sum\limits_{0}^{n} U_k$ - сходится $\Leftrightarrow$ $\{S_n\} $ --- ограничена
\begin{comment}
  Сходимость ряда эквивалентна ограниченности $S_n$
\end{comment}

\begin{theorem}
    $$U_k \geq 0, V_k \geq 0, \ \forall k:$$
    \begin{enumerate}
      \item
        Если $0 \leq U_k \leq V_k$, то если $\sum V_k$ сходится $\Rightarrow$
        $\sum U_k$ сходится и если
        $\sum U_k$ расходится $\Rightarrow$ $\sum V_k$ расходится.
      \item
        Если $\lim\limits_{n\to \infty} \frac{U_k}{V_k} = A > 0$, то ряды
        сходятся или расходятся одновременно.
    \end{enumerate}
\end{theorem}

\pagebreak
\begin{proof}
  \hfill
  \begin{enumerate}
    \item
      $\forall n$ верно неравенство $0 \leq \sum\limits_{0}^{n} U_k \leq
      \sum\limits_{0}^{n} V_k$
    \item
      $\forall \varepsilon > 0 \ \varepsilon < A \ \exists N : \forall n \geq N
      \Rightarrow 0 < A - \varepsilon < \frac{U_k}{V_k} < A + \varepsilon$ \\
      $0 < (A - \varepsilon) \cdot V_k < U_k < (A + \varepsilon) \cdot V_k$ \\
      Пусть $U_k$ --- сходится, тогда из доказанного выше 1ого пункта следует
      $(A - \varepsilon) \cdot V_k$ --- сходится
      $\Rightarrow \sum V_k$ сходится
      $\Rightarrow \sum (A + \varepsilon) \cdot V_k$ сходится
      $\Rightarrow \sum U_k$ сходится.
  \end{enumerate}
\end{proof}

\begin{remark}
  Вместо существования предела $\lim\limits_{n \to \infty} \frac{U_k}{V_k}$
  достаточно предположить, что существуют такие числа
  p и q > 0, такие что $0 < q < \frac{U_k}{V_k} < p, \ \forall k$
\end{remark}

\begin{theorem}[Признак Даламбера]
  $$\sum U_k, \ U_k > 0$$
  \begin{enumerate}
    \item Если $\exists q$ такое что:
      $\ \forall k \ \frac{U_{k+1}}{U_k} < q < 1$ сходится
    \item Если $\exists \lim\limits_{n \to \infty} \frac{U_{k+1}}{U_k} = q$, то:
    \begin{itemize}
      \item при $q < 1$ сходимость
      \item при $q > 1$ расходимость
      \item при $q = 1$ неизвестно (нужно провести дополнительные исследования)
    \end{itemize}
  \end{enumerate}
\end{theorem}

\begin{proof}
  Идея докозательства - сравнение с геометрической прогрессией.
  \begin{enumerate}
    \item $k = 0, 1, \dots, n$;
      $U_k = U_0 \cdot \frac{U_1}{U_0} \frac{U_2}{U_1} \cdots \frac{U_k}{U_{k-1}}
      < U_0 \cdot q^k$
      \begin{comment}
        $\frac{U_k}{U_{k-1}} < q, \ \forall k$
      \end{comment}
      $q < 1$, тогда $\sum U_0 \cdot q^k$ --- сходящаяся геометрическая прогрессия. \\
      $U_k = U_0 \cdot \frac{U_1}{U_0}\frac{U_2}{U_1}\cdots\frac{U_k}{U_{k-1}}
      \geq U_0 > 0$
      \begin{comment}
        $\frac{U_k}{U_{k-1}} \geq 1, \ \forall k$
      \end{comment}
      $U_k \not \to 0 \Rightarrow$ не выполняется необходимое условие сходимости.
    \item Пусть $\lim\limits_{k \to \infty} \frac{U_{k+1}}{U_k} = q$ \\
      $\forall \varepsilon > 0, \ \exists K : \ \forall k \geq K$
      выполняется неравенство
      $q - \varepsilon < \frac{U_{k+1}}{U_k} < q + \varepsilon$
      \begin{itemize}
        \item Если $q < 1$, то $(q + \varepsilon) \in [q, 1]$.
          Выберем такое $\varepsilon$, что $q + \varepsilon < 1$, для
          $\forall k \geq K(\varepsilon)$. \\
          $\frac{U_{k+1}}{U_k} < q + \varepsilon < 1 \Rightarrow$
          сходится по первой части.
        \item Если $q > 1$, то $(q - \varepsilon) \in [1, q]$.
          Выберем $\varepsilon$ так, чтобы
          $q - \varepsilon > 1$, для
          $\forall k \geq K(\varepsilon)$. \\
          $\frac{U_{k+1}}{U_k} > q - \varepsilon > 1, \ \Rightarrow$
          расходится по первой части.
      \end{itemize}
    \end{enumerate}
\end{proof}

\pagebreak
\begin{theorem}[Признак Коши]
  $$\sum U_k, U_k \geq 0$$
  \begin{enumerate}
    \item Если $\exists q < 1$ и $\forall k > K : $ выполняется
      $\sqrt[k]U_k \leq q < 1$, то ряд сходится, а если
      $\forall k \ \sqrt[k]U_k~\geq~1$, то расходится.
    \item Если $\exists \lim\limits_{k \to \infty} \sqrt[k]U_k = q, (q \geq 0)$,
      то
      \begin{itemize}
        \item $q < 1$ - сходится
        \item $q > 1$ - расходится
        \item $q = 1$ - нужны дополнительные исследования
      \end{itemize}
  \end{enumerate}
\end{theorem}
\begin{remark}
  $\lim\limits_{n \to \infty} \sqrt[n]U_n$ можно рассматривать вместо
  $\overline {\lim\limits_{k \to \infty}} \sqrt[k]U_k$
\end{remark}

\begin{proof}
  Сравнение с геометрической прогрессией
  \begin{enumerate}
    \item Если $\forall k\ \sqrt[k]U_k \leq q < 1 \Rightarrow U_k \leq q^k$
      --- сходящаяся геометрическая прогрессия. \\
      Если $\forall k \sqrt[k]U_k \geq 1 \Rightarrow U_k \geq 1$
      --- не выполняется необходимое условие сходимости.
    \item Если $\lim\limits_{k \to \infty} \sqrt[k]U_k = q$, то
      $\forall \varepsilon > 0 \ \exists K = K(\varepsilon) :
      \forall k \geq K, (q - \varepsilon) < \sqrt[k]U_k < (q + \varepsilon)$\\
      $(q - \varepsilon)^k < U_k < (q + \varepsilon)^k$
      \begin{itemize}
        \item При $q < 1$ выберем $\varepsilon$ так, чтобы $q + \varepsilon < 1$,
          тогда $U_k < (q + \varepsilon)^k < 1$
          --- сходящаяся геометрическая прогрессия.
        \item При $q > 1$ выберем $\varepsilon$ так, чтобы $q - \varepsilon > 1$,
          тогда $U_k > (q - \varepsilon)^k > 1$
          --- не выполняется необходимое условие сходимости.
      \end{itemize}
  \end{enumerate}
\end{proof}

\begin{definition}
  Дана $\{a_n\}$ и пусть $\overline {\lim\limits_{n\to \infty}} a_n$
  --- наибольший из частичных пределов, тогда:
  $$\forall \{a\} \ \exists \overline{\lim\limits_{n \to \infty}} a_n = A \ or \ \infty $$
  \begin{comment}
    $A$ --- число.
  \end{comment}
  \begin{itemize}
    \item Если $\overline{\lim\limits_{n \to \infty}} a_n = +\infty \Rightarrow
      \{a_n\}$ --- неограничена сверху $\Rightarrow
      \overline{\lim\limits_{k \to \infty}} \sqrt[k]U_k = +\infty$
      --- неограничена сверху. $U_k$ неограничена сверху и не выполняется
      необходимое условие.
    \item Если $\overline{\lim\limits_{n \to \infty}} a_n = A$, тогда
      $\forall \varepsilon \in (A - \varepsilon, A + \varepsilon)$ бесконечно
      много членов $\{a_n\}$: \\
      \begin{itemize}
        \item $\overline{\lim\limits_{k \to \infty}} \sqrt[k]U_k = q < 1$.
          Выберем $\varepsilon$ так, чтобы $q + \varepsilon < 1 \Rightarrow$
          $\exists K: \forall k \geq K, \ \sqrt[k]U_k < q + \varepsilon < 1$ по признаку
          Коши. \\
        \item $\overline{\lim\limits_{k \to \infty}} \sqrt[k]U_k = q > 1$.
          Выберем $\varepsilon$ так, чтобы $q - \varepsilon > 1 \Rightarrow$
          $\forall K \ \exists k \geq K: \sqrt[k]U_k~>~q~-~\varepsilon~>~1 \\
          \Rightarrow~U_k~>~1$
      \end{itemize}
  \end{itemize}
\end{definition}

\section{Интегральный признак сходимости рядов с неотрицательными членами}
\begin{theorem}
  Если $f(x)$ не отрицательна и убывает на $x \geq 1$, то ряд
  \begin{gather}
    \label{th131:series1}
    \sum\limits_{n = 1}^{\infty} f(n) < \infty
  \end{gather}
  сходится тогда и только тогда, когда сходится интеграл:
  \begin{gather}
    \label{th131:integral1}
    \int\limits_1^{+\infty} f(x) dx
  \end{gather}
  то есть $ \int\limits_1^{+\infty} f(x) dx < \infty$. \\
  \begin{itemize}
    \item $\sum a_n < \infty$ --- сходится
    \item $\sum a_n = \infty$ --- расходится
  \end{itemize}
\end{theorem}

\begin{proof}
  Если $k \leq x \leq k + 1, \ k = 1, 2, \dots,$ то, в силу убывания функции
  получаем неравенство:
  $$f(k) \geq f(x) \geq f(k + 1)$$
  Интегрируя по отрезку $[k, k + 1]$ получим:
  $$f(k) \geq \int\limits_k^{k+1} f(x) dx \geq f(k+1), \ k = 1, 2, \dots \ .$$

  \begin{gather}
    \label{th131:series2}
    \sum\limits_{k = 1}^{n} f(k+1) \leq \int\limits_1^{n+1} f(x) dx \leq
    \sum\limits_{k = 1}^{n} f(k)
  \end{gather}
  Пусть $S_n = \sum\limits_{k = 1}^{n} f(k),$ тогда \eqref{th131:series2} примет вид:
  \begin{gather}
    \label{th131:series3}
    S_n + 1 - f(1) \leq \int\limits_1^{n+1} f(x) dx \leq S_n
  \end{gather}
  Если ряд \eqref{th131:series1} сходится и его сумма равна $S$, то $S_n \leq S$,
  и $\int\limits_1^{n+1} f(x) dx \leq S, \ \forall n \in \mathbb{N}$. \\
  $\forall b > 1, n + 1 > b$ имеем:
  $$\int\limits_1^b f(x) dx \leq \int\limits_1^{n+1} f(x) dx \leq S$$
  В силу неотрицательности функции $f(x)$ интеграл сходится. \\
  Пусть наоборот, интеграл \eqref{th131:integral1} сходится, тогда из
  \eqref{th131:series3} следует:
  $$S_{n+1} \leq f(1) + \int\limits_1^{n+1} f(x) dx
  \leq f(1) + \int\limits_1^\infty f(x) dx$$
  Тем самым, последовательность сумм $\{S_n\}$ ряда \eqref{th131:series1}
  ограничена сверху, и поэтому этот ряд сходится.
\end{proof}

\begin{example}
  \label{th131:example131}
  \begin{gather}
    \label{ex131:series1}
    \sum\limits_{n = 1}^{\infty} \frac{1}{n^\alpha}, \alpha \in \mathbb{R}
  \end{gather}
  Положим $f(x) = \frac{1}{x^\alpha}$, тогда $f(n) = \frac{1}{n^\alpha}$ \\
  Поскольку $\int\limits_1^{+\infty} \frac{dx}{x^\alpha}$:
  \begin{itemize}
    \item При $\alpha > 1$ сходится
    \item При $\alpha \leq 1$ расходится
  \end{itemize}
  Тогда ряд \eqref{ex131:series1} сходится тогда и только тогда, когда $\alpha > 1$.
  При $\alpha < 0 \ $ дробь $ \ \frac{1}{n^\alpha} \geq 1$.
\end{example}

\section{Признак сходимости для чередующихся рядов}
Рассмотрим ряды с действительными числами, которые то положительные, то
отрицательные.

\begin{theorem}[Лейбница]
  Если
  \begin{gather}
    \lim\limits_{n \to \infty} U_n = 0 \label{th141:lim1} \\
    U_n \geq U_{n+1} > 0, \ n = 1, 2, \dots \label{th141:ineq1}
  \end{gather}
  то знакочередеющийся ряд
  \begin{gather}
    \sum\limits_{n = 1}^{+\infty} {(-1)}^{n+1} U_n \label{th141:sum1}
  \end{gather}
  сходится, при этом если $S$ --- сумма ряда, а $S_n$ --- его $n$-ая частичная
  сумма,\\ то $\forall n: n = 1, 2, \dots$
  \begin{gather}
    |S - S_n| \leq U_{n+1} \label{th141:ineq2}
  \end{gather}
\end{theorem}

\begin{proof}
  Заметим, что частичная суммы $S_n$ с четными номерами возрастают:
  $$S_{2k} = (U_1 - U_2) + (U_3 - U_4) + \dots + (U_{2k - 1} - U_{2k}), \
  k = 1, 2, \dots $$
  Так что выполняется неравенство $S_{2k + 2} \geq S_k$. Кроме того, они
  ограничены сверху:
  $$S_{2k} = U_1 - (U_2 - U_3) - \dots - (U_{2k - 2} - U_{2k - 1}) - U_{2k}, \
  S_{2k} < U_1$$
  Поэтому последовательность $\{S_{2k}\}$ сходится
  \begin{gather}
    \lim\limits_{k \to \infty} S_{2k} = S \label{th141:lim2}
  \end{gather}
  Поскольку $S_{2k + 1} = S_{2k} + U_{2k + 1}$ и $U_{2k + 1} \to 0$ при
  $k \to \infty$, то
  \begin{gather}
    \lim\limits_{k \to \infty} S_{2k + 1} = S \label{th141:lim3}
  \end{gather}
  Из \eqref{th141:lim2} и \eqref{th141:lim3} следует, что
  $\lim\limits_{n \to \infty} S_n = S$ \\
  При этом, нетрудно увидеть, что
  \begin{gather}
    S_{2k} \leq S \leq S_{2k + 1} \leq S_{2k - 1}, \ \forall k \label{th141:ineq3}
  \end{gather}
  Из неравенства \eqref{th141:ineq3} следует, что
  \begin{align*}
    S - S_{2k} \leq S_{2k + 1} - S_{2k} &= U_{2k + 1} \\
    S_{2k - 1} - S \leq S_{2k - 1} - S_{2k} &= U_{2k}, \ k = 1, 2, \dots
  \end{align*}
  Это и означает, что $\forall n \in \mathbb{N}$ выполняется
  неравенство \eqref{th141:ineq2}.
\end{proof}

\section{Преобразование Абеля}
\begin{theorem}
  Пусть $a_k \in \mathbb{C}$, $b_k \in \mathbb{C}$, $k = 1, \dots, n$;
  $B_k = b_1 + \dots + b_k$, тогда
  \begin{gather}
    \sum\limits_{k = 1}^{n} a_k b_k
  = \sum\limits_{k = 1}^{n - 1} (a_k - a_{k+1}) b_k + a_n B_n
  \label{th151:sum1}
  \end{gather}
\end{theorem}

\begin{proof}
  Очевидно, $b_1 = B_1, b_k = B_k - B_{k - 1}, k = 2, 3, \dots, n$ \\
  Поэтому
  $a_1 b_1 + a_2 b_2 + \dots + a_n b_n
  = a_1 B_1 + a_2 (B_2 - B_1) + a_3 (B_3 - B_2) + \dots a_n (B_n - B_{n - 1})
  = (a_1 - a_2) B_1 + (a_2 - a_3) B_2 + \dots + (a_{n-1} - a_n) B_{n-1} + a_n B_n$ \\
  Называется преобразованием Абеля $\sum\limits_{k = 1}^{n} a_k b_k$.
\end{proof}

\begin{consequence}[лемма Абеля]
  \label{th151:cons}
  Если $a_1 \leq a_2 \leq \dots \leq a_n$
   или $a_1 \geq a_2 \geq \dots \geq a_n$ \\
  $a_k \in \mathbb{R}, \ \forall k = 1, 2, \dots, n$,
  $|b_1 + \dots + b_k| \leq B, \ (b_k \in \mathbb{C})$, то
  $$\left|\sum\limits_{k = 1}^{n} a_k b_k \right| \leq B(|a_1| + 2|a_n|)$$
\end{consequence}
\begin{proof}
  $\left|\sum\limits_{k = 1}^{n} a_k b_k \right | \leq
  \sum\limits_{k = 1}^{n - 1} |a_k - a_{k + 1}| |B_k| + |a_n B_n| \leq
  B\left(\sum\limits_{k = 1}^{n - 1} |a_k - a_{k + 1}| + |a_n|\right) = \\
= B\left(\left| \sum\limits_{k = 1}^{n - 1} (a_k - a_{k + 1}) \right | + |a_n| \right)
= B(|a_1 - a_n| + |a_n|) \leq B(|a_1| + 2|a_n|).$
\end{proof}

\section{Признаки Дирихле и Абеля}
\begin{theorem}[признак Дирихле]
  \label{th161}
  Пусть дан ряд
  \begin{gather}
    \sum\limits_{n = 1}^{\infty} a_n b_n \label{th161:series1}
  \end{gather}
  \begin{enumerate}
    \item $a_n \in \mathbb{R}^n, b_n \in \mathbb{C}, \ n = 1, 2, \dots$
    \item $\{a_n\}, \{a_n\} \downarrow 0 \ (\{a_n\} \uparrow 0)$
    \item $\{B_n\}$ --- последовательность частичных сумм ряда $\sum b_n$ ограничена
  \end{enumerate}
  Тогда ряд \eqref{th161:series1} сходится.
\end{theorem}

\begin{proof}
  $\exists B > 0, \ |B_n| \leq B \ \forall n \Rightarrow \forall m \geq n \geq 2:
  |b_n + \dots + b_m| = |B_m - B_{n - 1}| \leq 2B$ \\
  Возьмем $\varepsilon > 0$. По скольку $a_n \to 0$, то
  $\exists N = N(\varepsilon): \forall n > N(\varepsilon)$ имеем
  $|a_n| < \frac{\varepsilon}{6B}$.\\
  Поэтому, $\forall n > N(\varepsilon)$ и $\forall m \geq n$ получим:
  $$|a_n b_n + \dots + a_m b_m| \leq 2B(|a_n| + 2|a_m|) <
  2B\left(\frac{\varepsilon}{6B} + 2 \frac{\varepsilon}{6B} \right) = \varepsilon$$
  Ряд \eqref{th161:series1} удовлетворяет Критерию Коши сходимости рядов.
\end{proof}
\begin{remark}
  Признак Лейбница - это частный случай признака Дирихле. \\
\end{remark}

\begin{theorem}[признак Абеля]
  \label{th162}
  Если последовательность действительных чисел $a_n$ монотонна и ограничена,
  ряд $\sum\limits_{n = 1}^{\infty} b_n, b_n \in \mathbb{C}$ сходится, то ряд
  \eqref{th161:series1} также сходится.
\end{theorem}

\begin{proof}
  $a_n = a + \alpha_n, \{\alpha_n\}$ --- монотонно стремящаяся к нулю
  последовательность. \\
  Поэтому
  $$\sum a_n b_n = \sum (a + \alpha_n)b_n = a \sum b_n + \sum \alpha_n b_n, $$
  где $a \sum b_n$ сходится по условию, а $\sum \alpha_n b_n$ сходится по
  признаку Дирихле. \\
  $\{B_n\}$ --- последовательность частичных сумм $\sum b_n$ ограничена,
  $\{\alpha_n\}$ --- монотонно стремящаяся к нулю последовательность.
\end{proof}

\section{Безусловно и условно сходящиеся ряды}
\begin{definition}
  Пусть $\{k_n\}, n = 1, 2, \dots$ --- последовательность, в которой каждое
  натуральное число встречается только один раз. $\{k_n\}$ --- однозначное
  отображение $a_n^* = a_{k_n}, \ (n~=~1, 2, \dots)$. \\
  Будем говорить, что ряд $\sum a_n^*$ является перестановкой ряда $\sum a_n$.
\end{definition}
\begin{definition}
  Говорят, что $\sum a_n$ сходится безусловно, если каждая перестановка
  сходится.
\end{definition}
\begin{theorem}
  \label{th171}
  Ряд $\sum a_n, (a_n \in \mathbb{C})$ сходится безусловно тогда и только тогда,
  когда он сходится абсолютно.
\end{theorem}

\begin{proof}
  Достаточность. \\
  Если ряд $\sum a_n$ сходится абсолютно, то все его перестановки сходятся к
  одному и тому же числу --- сумме исходного ряда. \\
  Пусть $\sum a_n^*$ --- перестановка ряда $\sum a_n$. $S_n^*$ --- ее частичная
  сумма. \\
  По Коши: $\forall \varepsilon > 0 \exists N : m \geq n > N$
  \begin{gather}
    |a_n| + \dots + |a_m| < \varepsilon \label{th171:uneq1}
  \end{gather}
  Выберем $p$ так, чтобы все натуральные числа $1, 2, \dots, N$ содержались в
  множестве $k_1, k_2, \dots, k_p$ (смотри определение), тогда при $n > p$
  $a_1, \dots, a_N$ в разности $S_n - S_n^*$ уничтожаются, так что
  $|S_n - S_n^*| < \varepsilon$ в силу \eqref{th171:uneq1}. \\
  Значит $\{S_n^*\}$ сходится к тому же пределу, что и $\{S_n\}$.
\end{proof}

\begin{definition}
  Сходящийся, но не абсолютно сходящийся ряд называется условно сходящимся.
\end{definition}
Из теоремы \eqref{th171} (из необходимости условия) $\Rightarrow$ Теорема \eqref{th172}
\begin{theorem}
  \label{th172}
  Условно сходящийся ряд не может сходится безусловно, то есть у него всегда
  существует расходящаяся перестановка.
\end{theorem}
\begin{proof}
  Без Доказательства.
\end{proof}

\begin{theorem}[Римана]
  Если ряд с действительными членами условно сходится, то каким бы не было
  действительное число $S$, существует перестановка ряда такая, что ее сумма
  равна $S$
\end{theorem}

\begin{proof}
  Без Доказательства.
\end{proof}
