\chapter{Числовые ряды}
\section{Определение}

\begin{definition}
  $U_1 + U_2 + U_3 + \dots = \sum\limits_{k = 0}^{\infty}{U_k}$
\end{definition}

\begin{definition}[Частичная сумма]
  $S_n = \sum\limits_{k = 0}^n U_k$
\end{definition}

\begin{definition}
  Ряд сходится, если $\exists \lim\limits_{n \to \infty}
                      \sum\limits_0^\infty U_k = S_n$
\end{definition}

\begin{definition}
  $\{a_n\} = a_0 + \sum\limits_1^n (a_k - a_{k-1})$, где $(a_k - a_{k - 1}) = U_k$
\end{definition}

\begin{theorem}[Критерий Коши]
  Ряд сходится, тогда и только тогда, когда он удовлетворяет условию Коши: \\
  $$\forall \varepsilon > 0, \ \exists N~=~N(\varepsilon), \
  \forall n \geq N, \forall p: \ |\sum\limits_{k=n+1}^{n+p} U_k|~=~|U_{n+1}
  + \dots + U_{n+p}|~=~|S_{n+p} - S_{n}| < \varepsilon$$
\end{theorem}

\begin{proof}
  $\sum U_k$ - сходится $\Leftrightarrow \{S_n\}$
  $$\forall \varepsilon > 0, \ \exists N: \forall n \geq N, \forall p: \
  |S_{n+p} - S_{n}| < \varepsilon$$
\end{proof}
\begin{consequence}[Необходимое условие сходимости]
  Если $\sum U_k$ сходится, то $U_k \to 0$, при $k~\to~\infty$
\end{consequence}
\begin{proof}
  Если $\sum U_k$ сходится, то выполняется Критерий Коши. При $p = 1$
  $$\forall \varepsilon > 0, \exists N, \forall n \geq N:
  |S_{n+1} - S_n|~=~|U_{n+1}| < \varepsilon$$
\end{proof}
\begin{consequence}
  Отбрасывание или добавление любого конечного числа членов ряда на его
  сходимость не влияет.
\end{consequence}

\begin{example}
  $\sum\limits_0^\infty z^n,$
  $S_n(z) = \sum\limits_0^n z_n = \frac{1-z^{n+1}}{1-z}$.
  При $n \to \infty, \ S_n(z) = \frac{1}{1 - z}, \ |z| < 1$. \\
  $S_n(z)$ не имеет придела при $|z| \geq 1$.
\end{example}

\section{Действия с рядами}

\begin{theorem}
  Ряды $\sum U_k$ и $\sum V_k$ сходятся к $\alpha$, тогда
  \begin{align}
    \label{series:property1}
    \sum\alpha U_k &= \alpha\sum U_k \\
    \label{series:property2}
    \sum (U_k \pm V_k) &= \sum U_k \pm \sum V_k
  \end{align}
\end{theorem}
\begin{proof}
  Доказательство свойства (\ref{series:property1}):
  $$\sum\limits_{k=0}^\infty \alpha U_k = \lim\limits_{n \to \infty}
  \sum\limits_0^n \alpha U_k = \alpha \lim\limits_{n \to \infty} \sum\limits_0^n U_k
  = \alpha \sum\limits_{k = 0}^\infty U_k$$
\end{proof}
\begin{proof}
  Доказательство свойства (\ref{series:property2}):
  \begin{gather*}
      \sum\limits_0^\infty U_k \pm \sum\limits_{0}^{\infty} V_k
      = \lim\limits_{n \to \infty} \sum\limits_{0}^{n} U_k
      \pm \lim\limits_{n \to \infty} \sum\limits_{0}^{n} V_k = \\
      = \lim\limits_{n \to \infty} \sum\limits_{0}^{n} U_k
      \pm \sum\limits_{0}^{n} V_k
      = \lim\limits_{n \to \infty} \sum\limits_{0}^{n} (U_k \pm V_k)
      = \sum\limits_{0}^{n} (U_k \pm V_k)
  \end{gather*}
\end{proof}

\begin{remark}
  Из сходимости $\sum (U_k \pm V_k) \not \Rightarrow$ сходимость
  $\sum U_k$ и $\sum V_k$
\end{remark}
\begin{remark}
  Если $\sum U_k$ сходится, то можно группировать, не меняя порядка.
\end{remark}
\begin{example}
  \begin{gather*}
    \sum (1 - 1) \\
    (1 - 1) + (1 - 1) + \dots \\
    1 - (1 - 1) - (1 - 1) \dots
  \end{gather*}
\end{example}
\begin{comment}
  Нельзя раскрывать скобки и переставлять члены.
\end{comment}

\subsection{Ряды с неотрицательными членами}
$U_k \geq 0$, $S_n = \sum\limits_{0}^{n} U_k$ - не убывающая последовательность. \\
$\sum\limits_{0}^{n} U_k$ - сходится $\Leftrightarrow$ $\{S_n\} $ --- ограничена
\begin{comment}
  Сходимость ряда эквивалентна ограниченности $S_n$
\end{comment}

\begin{theorem}
    $$U_k \geq 0, V_k \geq 0, \ \forall k:$$
    \begin{enumerate}
      \item
        Если $0 \leq U_k \leq V_k$, то если $\sum V_k$ сходится $\Rightarrow$
        $\sum U_k$ сходится и если
        $\sum U_k$ расходится $\Rightarrow$ $\sum V_k$ расходится.
      \item
        Если $\lim\limits_{n\to \infty} \frac{U_k}{V_k} = A > 0$, то ряды
        сходятся или расходятся одновременно.
    \end{enumerate}
\end{theorem}

\pagebreak
\begin{proof}
  \hfill
  \begin{enumerate}
    \item
      $\forall n$ верно неравенство $0 \leq \sum\limits_{0}^{n} U_k \leq
      \sum\limits_{0}^{n} V_k$
    \item
      $\forall \varepsilon > 0 \ \varepsilon < A \ \exists N : \forall n \geq N
      \Rightarrow 0 < A - \varepsilon < \frac{U_k}{V_k} < A + \varepsilon$ \\
      $0 < (A - \varepsilon) \cdot V_k < U_k < (A + \varepsilon) \cdot V_k$ \\
      Пусть $U_k$ --- сходится, тогда из доказанного выше 1ого пункта следует
      $(A - \varepsilon) \cdot V_k$ --- сходится
      $\Rightarrow \sum V_k$ сходится
      $\Rightarrow \sum (A + \varepsilon) \cdot V_k$ сходится
      $\Rightarrow \sum U_k$ сходится.
  \end{enumerate}
\end{proof}

\begin{remark}
  Вместо существования предела $\lim\limits_{n \to \infty} \frac{U_k}{V_k}$
  достаточно предположить, что существуют такие числа
  p и q > 0, такие что $0 < q < \frac{U_k}{V_k} < p, \ \forall k$
\end{remark}

\begin{theorem}[Признак Даламбера]
  $$\sum U_k, \ U_k > 0$$
  \begin{enumerate}
    \item Если $\exists q$ такое что:
      $\ \forall k \ \frac{U_{k+1}}{U_k} < q < 1$ сходится
    \item Если $\exists \lim\limits_{n \to \infty} \frac{U_{k+1}}{U_k} = q$, то:
    \begin{itemize}
      \item при $q < 1$ сходимость
      \item при $q > 1$ расходимость
      \item при $q = 1$ неизвестно (нужно провести дополнительные исследования)
    \end{itemize}
  \end{enumerate}
\end{theorem}

\begin{proof}
  Идея докозательства - сравнение с геометрической прогрессией.
  \begin{enumerate}
    \item $k = 0, 1, \dots, n$;
      $U_k = U_0 \cdot \frac{U_1}{U_0} \frac{U_2}{U_1} \cdots \frac{U_k}{U_{k-1}}
      < U_0 \cdot q^k$
      \begin{comment}
        $\frac{U_k}{U_{k-1}} < q, \ \forall k$
      \end{comment}
      $q < 1$, тогда $\sum U_0 \cdot q^k$ --- сходящаяся геометрическая прогрессия. \\
      $U_k = U_0 \cdot \frac{U_1}{U_0}\frac{U_2}{U_1}\cdots\frac{U_k}{U_{k-1}}
      \geq U_0 > 0$
      \begin{comment}
        $\frac{U_k}{U_{k-1}} \geq 1, \ \forall k$
      \end{comment}
      $U_k \not \to 0 \Rightarrow$ не выполняется необходимое условие сходимости.
    \item Пусть $\lim\limits_{k \to \infty} \frac{U_{k+1}}{U_k} = q$ \\
      $\forall \varepsilon > 0, \ \exists K : \ \forall k \geq K$
      выполняется неравенство
      $q - \varepsilon < \frac{U_{k+1}}{U_k} < q + \varepsilon$
      \begin{itemize}
        \item Если $q < 1$, то $(q + \varepsilon) \in [q, 1]$.
          Выберем такое $\varepsilon$, что $q + \varepsilon < 1$, для
          $\forall k \geq K(\varepsilon)$. \\
          $\frac{U_{k+1}}{U_k} < q + \varepsilon < 1 \Rightarrow$
          сходится по первой части.
        \item Если $q > 1$, то $(q - \varepsilon) \in [1, q]$.
          Выберем $\varepsilon$ так, чтобы
          $q - \varepsilon > 1$, для
          $\forall k \geq K(\varepsilon)$. \\
          $\frac{U_{k+1}}{U_k} > q - \varepsilon > 1, \ \Rightarrow$
          расходится по первой части.
      \end{itemize}
    \end{enumerate}
\end{proof}

\pagebreak
\begin{theorem}[Признак Коши]
  $$\sum U_k, U_k \geq 0$$
  \begin{enumerate}
    \item Если $\exists q < 1$ и $\forall k > K : $ выполняется
      $\sqrt[k]U_k \leq q < 1$, то ряд сходится, а если
      $\forall k \ \sqrt[k]U_k~\geq~1$, то расходится.
    \item Если $\exists \lim\limits_{k \to \infty} \sqrt[k]U_k = q, (q \geq 0)$,
      то
      \begin{itemize}
        \item $q < 1$ - сходится
        \item $q > 1$ - расходится
        \item $q = 1$ - нужны дополнительные исследования
      \end{itemize}
  \end{enumerate}
\end{theorem}
\begin{remark}
  $\lim\limits_{n \to \infty} \sqrt[n]U_n$ можно рассматривать вместо
  $\overline {\lim\limits_{k \to \infty}} \sqrt[k]U_k$
\end{remark}

\begin{proof}
  Сравнение с геометрической прогрессией
  \begin{enumerate}
    \item Если $\forall k\ \sqrt[k]U_k \leq q < 1 \Rightarrow U_k \leq q^k$
      --- сходящаяся геометрическая прогрессия. \\
      Если $\forall k \sqrt[k]U_k \geq 1 \Rightarrow U_k \geq 1$
      --- не выполняется необходимое условие сходимости.
    \item Если $\lim\limits_{k \to \infty} \sqrt[k]U_k = q$, то
      $\forall \varepsilon > 0 \ \exists K = K(\varepsilon) :
      \forall k \geq K, (q - \varepsilon) < \sqrt[k]U_k < (q + \varepsilon)$\\
      $(q - \varepsilon)^k < U_k < (q + \varepsilon)^k$
      \begin{itemize}
        \item При $q < 1$ выберем $\varepsilon$ так, чтобы $q + \varepsilon < 1$,
          тогда $U_k < (q + \varepsilon)^k < 1$
          --- сходящаяся геометрическая прогрессия.
        \item При $q > 1$ выберем $\varepsilon$ так, чтобы $q - \varepsilon > 1$,
          тогда $U_k > (q - \varepsilon)^k > 1$
          --- не выполняется необходимое условие сходимости.
      \end{itemize}
  \end{enumerate}
\end{proof}

\begin{definition}
  $\{a_n\}$ \\
  $\overline \lim_{n\to \infty} a_n$
\end{definition}

\begin{proof}[Признак Коши с верзним пределом]
  $\overline \lim_{k \to \infty} \sqrt[k]U_k = q < 1$
\end{proof}

\begin{remark}
  Признак Даламбера слабее признака Коши
\end{remark}
