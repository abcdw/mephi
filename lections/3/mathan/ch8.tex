\section{Определения}
Вместо терминов числовая функция в точке и вектор-функции точки будем
употреблять такие равнозначные к ним понятия: скалярное поле и векторное
поле(тем самым подчеркивается, что значение функции не зависит от выбора
системы координат).

\begin{definition}
  Пусть в области $\mathbb{G}$ задана $\pvec a = \pvec a(M)$ и существует $u =
  u(M), \ M \in \mathbb{G}$, такая что
  \begin{gather}
    \pvec a(M) = \mbox{grad} \ u(M)
    \label{def811:eq1}
  \end{gather}
  Тогда функция $u(M)$ называется потенциалом поля $A$. Поле обладающее
  потенциалом называется потенциальным полем.
\end{definition}

Введем $\nabla = \{\frac{\partial}{\partial x}, \frac{\partial}{\partial y},
\frac{\partial}{\partial z}\}$ (оператор Набла). Тогда $\nabla U =
\{\frac{\partial U}{\partial x}, \frac{\partial U}{\partial y},
\frac{\partial U}{\partial z}\}$ и равенство \eqref{def811:eq1} можно записать
в виде $\pvec a(M) = \nabla U$.

\begin{definition}
  Пусть поле $\pvec a = \{P, Q, R\}$ дифференцируема в области $\mathbb{G}$,
  числовая функция
  \begin{gather*}
    \mbox{div} \ \pvec a = (\nabla, \pvec a) = \frac{\partial P}{\partial x} +
    \frac{\partial Q}{\partial y} + \frac{\partial R}{\partial z}
  \end{gather*}
  называется дивиргенцией поля $\pvec a$ в области $\mathbb{G}$.
  Векторная функция
  \begin{gather*}
    \mbox{rot} \ \pvec a = [\pvec \nabla, \pvec a] =
    \begin{vmatrix}
      \pvec i & \pvec j & \pvec k \\
      \frac{\partial }{\partial x} & \frac{\partial }{\partial y} &
      \frac{\partial }{\partial z} \\
      P & Q & R
    \end{vmatrix} =
    \left\{\frac{\partial R}{\partial y} - \frac{\partial Q}{\partial z},
    \frac{\partial P}{\partial z} - \frac{\partial R}{\partial x},
    \frac{\partial Q}{\partial x} - \frac{\partial P}{\partial y}\right\}
  \end{gather*}
  называется ротором или вихрем поля A.
\end{definition}

\begin{definition}
  Если векторное поле $\pvec a$ задано на кусочно-гладкой замкнутой кривой~
  $\Gamma$,~то
  \begin{gather*}
    \oint\limits_{\Gamma} \pvec a d \pvec r
  \end{gather*}
  называется циркуляцией
  векторного поля $\pvec a$ по кривой $\Gamma$.
\end{definition}

\begin{definition}
  Если кусочно-гладкая поверхность $S$ ориентирована с помощью единичной
  нормали $\pvec \nu$, то для векторного поля $\pvec a$, заданного на
  поверхности $S$:
  \begin{gather*}
    \iint\limits_S (\pvec a, \pvec \nu) dS
  \end{gather*}
  называется потоком векторного поля, через поверхность $S$.
\end{definition}

\section{Формула Гаусса-Остроградского}
\begin{theorem}
  \label{th821}
  Пусть $\mathbb{G}$ --- ограниченная область $\subset \mathbb{R}^3_{xyz}$
  с кусочно-гладкой границей $\partial G$, $P, Q, R$ --- функции, непрерывные
  вместе со своими частными производными: \\ $\frac{\partial P}{\partial x},
  \frac{\partial Q}{\partial y}, \frac{\partial R}{\partial z}, \ (x, y, z) \in
  \overline{\mathbb{G}}$. $\pvec a = \{P, Q, R\}$, тогда
  \begin{gather}
    \iiint\limits_{\mathbb{G}} \mbox{div} \ \pvec a dx dy dz =
    \iint\limits_{\partial \mathbb{G}} (\pvec a, \pvec \nu) dS
    \label{th821:eq1}
  \end{gather}
  где $\pvec \nu$ --- единичная нормаль к $\partial \mathbb{G}$, внешняя
  относительно области $\mathbb{G}$. То есть поток векторного поля через
  границу области равен интегралу от дивиргенции поля по самой области.
  \eqref{th821:eq1} называется формулой Остроградского-Гаусса.
  \begin{gather}
    \iiint \left( \frac{\partial P}{\partial x} + \frac{\partial Q}{\partial y}
    + \frac{\partial R}{\partial z}\right) dx dy dz = \iint\limits_{\partial
      \mathbb{G}^+} P dy dz + Q dx dz + R dx dy
    \label{th821:eq2}
  \end{gather}
  где $\partial \mathbb{G}^+$ ориентирована внешней нормалью $\pvec \nu$.
\end{theorem}

\begin{proof}
  $\mathbb{R}^3$
  \begin{definition}
    $\mathbb{G} \subset \mathbb{R}^3, \mathbb{G}_z, S_0$ с образующими
    параллельными оси $Oz$, $S_1, S_2, \varphi, \psi, \\ \mathbb{D} \subset
    \mathbb{R}^2$ с кусочно-гладкой границей $\partial \mathbb{D}$.
    $\mathbb{G}_z = \{(x, y, z) : (x, y) \in \mathbb{D}, \varphi(x, y) < z <
    \psi(x, y)\}$. Аналогично определяются области $\mathbb{G}_y$ и
    $\mathbb{G}_x$, цилиндрической поверхности, образующие которой параллельны
    $Oy$ и $Ox$. Простой областью называется область, допускающая разбиение на
    области каждого из типов: $\mathbb{G}_x, \mathbb{G}_y, \mathbb{G}_z$.
  \end{definition}
\end{proof}

\begin{lemma}
  Пусть в теореме \eqref{th821} $\mathbb{G}$ --- область типа $\mathbb{G}_z$,
  тогда
  \begin{gather}
    \iiint\limits_{\mathbb{G}} \frac{\partial R}{\partial z} dx dy dz =
    \iint\limits_{\partial \mathbb{G}^+} R dx dy
    \label{lem821:eq1}
  \end{gather}
\end{lemma}

\begin{proof}
  \begin{gather}
    \iiint\limits_{\mathbb{G}} \frac{\partial R}{\partial z} dx dy dz =
    \iint\limits_{\mathbb{D}} dx dy \int\limits_{\varphi(x, y)}^{\psi(x, y)}
    \frac{\partial R}{\partial z} dz = \iint\limits_{\mathbb{D}} \left(R(x, y,
    \psi(x, y)) - R(x, y, \varphi(x, y)) \right) dx dy = \nonumber \\
    \iint\limits_{\mathbb{D}} R(x, y, \psi(x, y)) dx dy -
    \iint\limits_{\mathbb{D}} R(x, y, \varphi(x, y)) dx dy
    \overset{\eqref{ex761:eq2}} = \iint\limits_{S_2^+} R dx dy +
    \iint\limits_{S_1^-} R dx dy
    \label{lem821:eq2}
  \end{gather}
  где $S_2^+$ --- верхняя сторона поверхности $S_2$, а $S_1^-$ --- нижняя
  сторона поверхности $S_1$. \\ Поскольку двойной $\iint\limits_{S_0^+} R dx dy =
  0$ (см. \eqref{ex762:eq1}), $\partial \mathbb{G} = S_1 \cup S_2 \cup S_0$, то
  согласно \eqref{def762} из \eqref{lem821:eq2} получаем \eqref{lem821:eq1}.
\end{proof}

\begin{lemma}
  Если область $\mathbb{G}$ можно разбить на конечное число областей типа
  $\mathbb{G}_z$, то выполняется \eqref{lem821:eq1}.
\end{lemma}

\begin{proof}
  Очевидно, что на поверхности, по которой прилигают две такие области,
  индуцируются противоположные ориентации, поэтому при положительном
  интегрировании по границе, произойдут взаимные уничтожения, в результате
  которых останется лишь интеграл $\partial \mathbb{G}^+$ исходной области
  $\mathbb{G}$.
\end{proof}

Если область $\mathbb{G}$ можно разложить на области типа $\mathbb{G}_y,
\mathbb{G}_x$, то соответственно имеют место соотношения:
\begin{gather}
  \iiint\limits_{\mathbb{G}} \frac{\partial Q}{\partial y} dx dy dz =
  \iint\limits_{\mathbb{G}^+} Q dx dz
  \label{ch82:eq1} \\
  \iiint\limits_{\mathbb{G}} \frac{\partial Q}{\partial x} dy dz dx =
  \iint\limits_{\mathbb{G}^+} Q dy dz
  \label{ch82:eq2}
\end{gather}
если область $\mathbb{G}$ простая, то складывая \eqref{lem821:eq1},
\eqref{ch82:eq1} и \eqref{ch82:eq2}. Получим для области $\mathbb{G}$ равенство
\eqref{th821:eq2}.

\section{Формула Стокса}
\begin{theorem}
  $\pvec a$ непрерывно дифференцируемая вектор-функция в области $\mathbb{G}$,
  содержащей ориентированную кусочно-гладкую поверхность $S = S_1 \cup \dots
  \cup S_k, \ S_1, \dots, S_k$ --- гладкие куски поверхности и пусть $\partial
  S$ --- ее край с ориентацией, согласованной с заданной ориентацией
  поверхности $S$, тогда:
  \begin{gather}
    \int\limits_{\partial S} \pvec a d \pvec r = \iint\limits_S \mbox{rot} \
    \pvec a d \pvec S
    \label{th831:eq1}
  \end{gather}
  таким образом циркуляция векторного поля на границе поверхности равна потоку
  поля через поверхность. Наглядное согласование ориентации $\partial S$ с
  ориентацией поверхности означает следующие: наблюдатель, двигающийся по краю
  $\partial S$ (этот край может состоять из нескольких контуров), и, смотрящий
  из конца нормали $\pvec \nu$, видит непосредственно прилегающую к нему часть
  поверхности слева от себя ($\pvec \nu$ --- определяет ориентацию того куска
  $S_i$, на котором в данный момент находится наблюдатель). Формула
  \eqref{th831:eq1} называется формулой Стокса. Если $S \subset \mathbb{R}^2$,
  то получаем формулу Грина.
\end{theorem}

\begin{proof}
  Очевидно, достаточно провести доказательство для гладкой поверхности, так как
  написав формулу Стокса для каждой поверхности $S_i$ ($i = 1, \dots, k$), и
  положив получившиеся равенства получим формулу \eqref{th831:eq1}. \\

  Чтобы доказательство упростить проведем его с дополнительными условиями на
  гладкой поверхности $S : S = \{\pvec r(u, v): (u, v) \in
  \overline{\mathbb{D}}\}$ --- давжды непрерывно дифференцируемая
  ориентированная поверхность, без особых точек в пространстве
  $\mathbb{R}_{xyz}^3, \mathbb{D}$ --- плоская ограниченная область у которой
  область $\partial \mathbb{D}$ --- есть простой кусочно-гладкий контур. Пусть
  $u = u(t), v = v(t), \ t \in [a,b]$ --- параметрическое представление
  контуров $\partial {\mathbb{D}}^+, \partial S$ --- край поверхности $S$ с
  представлением: $\pvec r = \pvec r(u(t), v(t)), \ a \leq t \leq b, \ S^+$ ---
  поверхность $S$, оиентированная нормалью $\pvec \nu = \frac{(\pvec r_u',
  \pvec r_v')}{|(\pvec r_u', \pvec r_v')|} = \{\cos \alpha, \cos \beta, \cos
  \gamma\}$. \\
  Поскольку $(\mbox{rot} \ \pvec a, \pvec \nu) = (\pvec \nu, [\nabla, \pvec a])
  = \begin{vmatrix}
    \cos \alpha & \cos \beta & \cos \gamma \\
    \frac{\partial }{\partial x} & \frac{\partial }{\partial y} &
    \frac{\partial }{\partial z} \\
    P & Q & R
  \end{vmatrix}$, то в координатной записи формула \eqref{th831:eq1} примет
  вид:
  \begin{gather}
    \int\limits_{\partial S} Pdx + Qdy + Rdz = \iint\limits_{S^+}
    \left(\frac{\partial R}{\partial y} - \frac{\partial Q}{\partial z} \right)
    dy dz + \left(\frac{\partial P}{\partial z} - \frac{\partial R}{\partial x}
    \right) dx dz + \left(\frac{\partial Q}{\partial x} - \frac{\partial
    P}{\partial y} \right) dx dy
    \label{th831:eq2}
  \end{gather}
  поэтому достаточно проверить, что:
  \begin{gather*}
    \int\limits_{\partial S} Pdx = \iint\limits_{S^+} \frac{\partial
    P}{\partial z} dx dz - \frac{\partial P}{\partial y} dx dy \\
    \int\limits_{\partial S} Qdy = \iint\limits_{S^+} \frac{\partial
    Q}{\partial x} dx dy - \frac{\partial Q}{\partial z} dy dz \\
    \int\limits_{\partial S} Rdz = \iint\limits_{S^+} \frac{\partial
    R}{\partial y} dy dz - \frac{\partial R}{\partial x} dx dz \\
  \end{gather*}
  сложив эти формулы, получим формулу \eqref{th831:eq2}. Проверим пример
  формулы:
  \begin{gather*}
    \int\limits_{\partial S} P(x, y, z) dx = \int\limits_a^b P(\pvec r(u(t),
    v(t)) \cdot x_t' (u(t), v(t)) dt = \int\limits_a^b P(\pvec r(u(t), v(t)))
    \left(\frac{\partial x}{\partial u} \frac{\partial u}{\partial t} +
    \frac{\partial x}{\partial v} \frac{\partial v}{\partial t}\right) dt = \\
    \int\limits_{\partial \mathbb{D}^+} P(\pvec r(u, v)) \frac{\partial
    x}{\partial u} du + P(\pvec r(u, v)) \frac{\partial x}{\partial v} dv =
    \iint\limits_{\mathbb{D}} \left[ \frac{\partial }{\partial u}\left(P
      \frac{\partial x}{\partial v} \right) - \frac{\partial }{\partial
      v}\left(P \frac{\partial x}{\partial u} \right) \right] du dv =\\
      \iint\limits_{\mathbb{D}} \left[\left(\frac{\partial P}{\partial x}
        \frac{\partial x}{\partial u} + \frac{\partial P}{\partial y}
        \frac{\partial y}{\partial u} + \frac{\partial P}{\partial z}
        \frac{\partial z}{\partial u}\right) \frac{\partial x}{\partial v} +
        P \frac{\partial^2 x}{\partial v \partial u} - \left(\frac{\partial
        P}{\partial x} \frac{\partial x}{\partial v} + \frac{\partial
        P}{\partial y} \frac{\partial y}{\partial v} + \frac{\partial
        P}{\partial z} \frac{\partial z}{\partial v} \right) \frac{\partial
        x}{\partial u} - P\frac{\partial^2 x}{\partial u \partial v}\right] du
        dv = \\
        \iint\limits_{\mathbb{D}} \left[\frac{\partial P}{\partial z}
        \frac{\mathbb{D}(z, x)}{\mathbb{D}(u, v)} - \frac{\partial
        P}{\partial y}\frac{\mathbb{D}(x, y)}{\mathbb{D}(u, v)}\right] du dv =
        \iint\limits_{S^+} \frac{\partial P}{\partial z} dx dz -
        \iint\limits_{S^+} \frac{\partial P}{\partial y} dx dy
  \end{gather*}
\end{proof}

\section{Инвариантность понятий дивергенция и ротор}
Далее $\pvec a = \pvec a(M), \ M \in \mathbb{G}$ --- непрерывно
дифференцируемое векторное поле в трехмерной области $\mathbb{G}$.

\begin{theorem}
  Пусть $\mathbb{V} \subset \mathbb{G}$ --- окрестность (например шаровая).
  $|\mathbb{V}|$ --- ее объем. $d$ --- диаметр, тогда:
  \begin{gather}
    \mbox{div} \ \pvec a(M_0) = \lim\limits_{d \to 0}
    \frac{\int\limits_{\partial \mathbb{V}} (\pvec a, \pvec \nu)
    dS}{|\mathbb{V}|}
    \label{th841:eq1}
  \end{gather}
  $\pvec \nu$ --- внешняя нормаль к границе $\partial \mathbb{V}$ области
  $\mathbb{V}$.
\end{theorem}

\begin{proof}
  Используя формулу Остроградского-Гаусса и теорему о среднем получим:
  \begin{gather}
    |\mathbb{V}| \cdot \mbox{div} \ \pvec a(M') = \iint\limits_{\partial
      \mathbb{V}}
    (\pvec a, \pvec \nu) dS
    \label{th841:eq2}
  \end{gather}
  где $M' \in \mathbb{V}$. При $d \to 0: M' \to M_0, \ \mbox{div} \ M' \to
  \mbox{div} \ M_0$. Отсюда из \eqref{th841:eq2} следует \eqref{th841:eq1}.
\end{proof}
Правая часть формулы \eqref{th841:eq1} не зависит от выбора координат. Поэтому
$\mbox{div} \ \pvec a$ инвариантна. \\

Если считать $\pvec a$ полем скоростей течения жидкости или газа, то
дивиргенцию можно интерпретировать как плотность распределения источников в
плотности течения.

\begin{theorem}
  Пусть $M_0 \in \mathbb{G}, \pvec \nu$ --- произвольный фиксированный
  единичный вектор $\Pi$ --- плоскость, проходящая через точку $M_0$
  перпендикуляроно $\pvec \nu$. $S$ --- окрестность (например круговая) точки
  $M_0$ плоскости $\Pi$ ($S \subset \Pi \cap \mathbb{G}$), $|S|$ --- ее
  площадь, $d$ --- ее диаметр. И пусть контур $\pvec S$ согласованно ориентирован с
  нормалью $\pvec \nu$. Тогда:
  \begin{gather}
    (\mbox{rot} \ \pvec a(M_0), \pvec \nu) = \lim\limits_{d \to 0}
    \frac{\int\limits_{\partial S} \pvec a d \pvec r}{|S|}
    \label{th842:eq1}
  \end{gather}
\end{theorem}

\begin{proof}
  Из формулы Стокса и теоремы о среднем получим:
  \begin{gather}
    |S| \cdot (\mbox{rot} \ \pvec a(M'), \pvec \nu) = \int\limits_{\partial S}
    \pvec a d \pvec r, \ M' \in S
    \label{th842:eq2}
  \end{gather}
  если $d \to 0$, то $M' \to M_0, (\mbox{rot} \ \pvec a(M'), \pvec \nu) \to
  (\mbox{rot} \ \pvec a(M_0), \pvec \nu)$. Отсюда из \eqref{th842:eq2} получаем
  \eqref{th842:eq1}.
\end{proof}

Правая часть равенства \eqref{th842:eq1} не зависит от выбор системы координат.
Выбрав в качестве $\pvec \nu$ три линейно независимых вектора получим по
формуле \eqref{th842:eq1} три проекции $\mbox{rot} \ \pvec a$ на эти векторы.
Этими своими проекциями ротор однозначно определяется. Поскольку они не зависят
от выбора системы координат, следовательно $\mbox{rot} \ \pvec a$ не зависит от
системы координат.

\section{Потенциальные векторные поля}
Поверхность $S$, для которой
\begin{definition}
  \label{def851}
  Множество $\mathbb{E} \subset \mathbb{R}^3$ называется односвязанным, если
  для любого кусочно-гладкого замкнутого контура, принадлежащего $\mathbb{E}$,
  существует допустимая поверхность, краем которой он являлся и которая так же
  лежит в $\mathbb{E}$. \\
\end{definition}
Если $\mathbb{E}$ --- плоская область $\mathbb{G} \subset \mathbb{R}^2$, то
определение \eqref{def851} равносильно условию: для любого кусочно-гладкого
контура, ограниченная им облать $\mathbb{D}$ содержится в $\mathbb{G}$ (то есть
односвязная плоская область не имеет дыр). \\

В пространстве примером односвязных областей являются выпуклые области,
множество точек лежащих между двумя коцентрическими сферами.

\begin{theorem}
  Пусть в односвязной области $\mathbb{G} \subset \mathbb{R}^3$ задано
  непрерывно дифференцируемое векторное поле $\pvec a = \{P, Q, R\}$, тогда
  эквивалентны следующие пять свойств:
  \begin{enumerate}
    \item $\pvec a = \pvec a(M)$ --- потенциальна в $\mathbb{G}$
    \item $Pdx + Qdy + Rdz$ является в $\mathbb{G}$ полным дифференциалом
      некоторой функции $u = u(M)$, которая и является потенциалом поля $\pvec
      a$
    \item Для любой точки $A \in \mathbb{G}$ и $B \in \mathbb{G}
      \int\limits_{\wideparen{AB}} \pvec a d \pvec r$ не зависит от кривой
      $\wideparen{AB}$, соединяющей эти точки в области $\mathbb{G}$
      (предполагается, что $\wideparen{AB}$ --- простая кусочно-гладкая
      кривая)
    \item $\int \pvec a d \pvec r = 0$ для любого кусочно-гладкого контура
      $\Gamma \subset \mathbb{G}$
    \item $\mbox{rot} \ \pvec a = 0$ в области $\mathbb{G}$
  \end{enumerate}
\end{theorem}

\begin{proof}
  \begin{enumerate}
    \item Из 2 следует 1. \\

      Пусть в $\mathbb{G}$ существует функция $U(M) = U(x, y, z): dU = Pdx +
      Qdy + Rdz$. Тогда $P = \frac{\partial U}{\partial x}, Q = \frac{\partial
      U}{\partial y}, R = \frac{\partial U}{\partial z}$, так что $\pvec a =
      \mbox{grad} \ U$.
    \item Из 1 следует 5. \\

      Пусть $\pvec a = \mbox{grad} \ U = \nabla U$, тогда $\mbox{rot} \ \pvec a
      = [\nabla, \pvec a] = [\nabla, \nabla U] = \begin{vmatrix}
        \pvec i & \pvec j & \pvec k \\
        \frac{\partial }{\partial x} & \frac{\partial }{\partial y} &
        \frac{\partial }{\partial z} \\
        \frac{\partial U}{\partial x} & \frac{\partial U}{\partial y} &
        \frac{\partial U}{\partial z}
      \end{vmatrix} = 0$. Равенство смешанных производных обеспечено
      непрерывной дифференцируемостью функции $P, Q, R$.
    \item Из 5 следует 4. \\

      Пусть $\mbox{rot} \ \pvec a = 0$ в области $\mathbb{G}$. В силу
      односвязности области $\mathbb{G}$ существует допустимая поверхность $S
      \subset \mathbb{G}$ для которой $\Gamma$ является краем. Тогда по теореме
      Стокса:
      \begin{gather*}
        \int\limits_{\Gamma} \pvec a d \pvec r = \iint\limits_S \mbox{rot} \
        \pvec a d \pvec S = 0
      \end{gather*}
    \item Из 4 следует 3. \\

      Пусть $\int\limits_{\Gamma} \pvec a d \pvec r = 0, \ \forall \Gamma \in
      \mathbb{G}$ (кусочно-гладкой). $A \in \mathbb{G}, B \in \mathbb{G}, \
      (\wideparen{AB})_1, (\wideparen{AB})_2$ --- кусочно гладкие простые
      кривые, соединяющие в $\mathbb{G}$ точки $A, B$. Если эти кривые не имеют
      общих точек, то кривая $\Gamma := (\wideparen{AB})_1 \cup
      (\wideparen{AB})_2$ является простым кусочно-гладким контуром, лежащим в
      $\mathbb{G}$, тогда:
      \begin{gather*}
        \oint\limits_{\Gamma} = \int\limits_{(\wideparen{AB})_1} -
        \int\limits_{(\wideparen{AB})_2} = 0 \ \Longrightarrow
        \int\limits_{(\wideparen{AB})_1} = \int\limits_{(\wideparen{AB})_2}
      \end{gather*}
      если же $(\wideparen{AB})_1 \cap (\wideparen{AB})_2 \not = 0$, то в
      области $\mathbb{G}$ нужна третья кривая $(\wideparen{AB})_3$, которая не
      пересекается ни с одной из прежних. Тогда по доказанному:
      $\int\limits_{(\wideparen{AB})_1} = \int\limits_{(\wideparen{AB})_2} =
      \int\limits_{(\wideparen{AB})_3}$
    \item из 3 следует 2. \\

      Зафиксируем какую-либо точку $M_0 \in \mathbb{G}$ и определим функцию
      $U(M)$ по формуле:
      \begin{gather}
        U(M) = \int\limits_{\wideparen{M_0M}} \pvec a d \pvec r =
        \int\limits_{\wideparen{M_0M}} Pdx + Qdy + Rdz, \ M \in \mathbb{G}
        \label{th851:eq1}
      \end{gather}
      где $\wideparen{M_0M}$ --- какая-либо простая кусочно-гладкая кривая,
      соединяющая в $\mathbb{G}$ точки $M_0$ и $M$. Формула \eqref{th851:eq1} в
      силу свойства 3 определяет функцию $U(M)$ однозначно. Покажем что:
      \begin{gather}
        dU = Pdx + Qdy + Rdz
        \label{th851:eq2}
      \end{gather}
      в любой точке $M(x, y, z) \in \mathbb{G}$. Пусть $M'(x + h, y, z) \in
      \mathbb{G}, \wideparen{MM'}$ --- отрезок с концами $M$ и $M'$. Тогда:
      \begin{gather*}
        U(M') - U(M) = \int\limits_{\wideparen{MM'}} Pdx + Qdy + Rdz =
        \int\limits_x^{x + h} P(t, y, z) dt
      \end{gather*}
      откуда получаем:
      \begin{gather*}
        \frac{\partial U}{\partial x} = \lim\limits_{h \to 0} \frac{U(x + h, y,
        z) - U(x, y, z)}{h} = \lim\limits_{h \to 0} \frac{1}{h}
        \int\limits_x^{x+h} P(t, y, z) dt = P(x, y, z)
      \end{gather*}
      Аналогично получим $\frac{\partial U}{\partial y} = Q, \ \frac{\partial
      U}{\partial z} = R$. Поскольку функции $P, Q, R$ непрерывны, то функция
      $U(M)$ дифференцируема и равенство \eqref{th851:eq2} доказано.
  \end{enumerate}
\end{proof}

\begin{remark}
  Потенциал $U(M)$ поля $\pvec a$ определен с точностью до аддитивной
  постоянной. Из формулы \eqref{th851:eq1} следует, что потенциальное поле
  $\pvec a: \int \pvec a d \pvec r = U(B) - U(A)$.
\end{remark}

\begin{remark}
  Линиями потенциала называют $U, \pvec F = -\mbox{grad} \ U$.
\end{remark}

\section{Соленоидальные векторные поля}
\begin{definition}
  Непрерывная в области $\mathbb{G} \subset \mathbb{R}^3$ векторное поле $\pvec
  a$ называется соленоидальным, если для любой ограниченной области $\mathbb{V}
  \subset \mathbb{G}$ с кусочно-гладкой границей $\partial \mathbb{V} \subset
  \mathbb{G}$ его поток через эту границу равен 0.
  Очевидно, что понятия соленоидальности не зависит от выбора ориентации на
  границе $\partial \mathbb{V}$ обалсти $\mathbb{V}$.
\end{definition}

\begin{theorem}
  Для того, чтобы непрерывно дифференцируемое векторное поле $\pvec a$ было
  соленоидальным в области $\mathbb{G}$ необходимо и достаточно, чтобы на всех
  точках области выполнялось равенство:
  \begin{gather}
    \mbox{div} \ \pvec a = 0
    \label{th861:eq1}
  \end{gather}
\end{theorem}

\begin{proof}
  \underline{Необходимость}. Пусть $\pvec a$ соленоидально в $\mathbb{G}, M \in
  \mathbb{G}, r_0 \in \mathbb{G}$. Тогда $r_0 > 0, \forall r : r < r_0$ и с
  центром в точке $M$ содержится в $\mathbb{G}$. Для этих шаров:
  \begin{gather*}
    \iint\limits_{\partial \mathbb{V}_r} \pvec a d \pvec S = 0 \\
    \mbox{div} \ \pvec a(M) = \lim\limits_{r \to 0}
    \frac{\iint\limits_{\partial \mathbb{V}_r} \pvec a d \pvec
    S}{|\mathbb{V}_R|} = 0
  \end{gather*} \\

  \underline{Достаточность}. Если выполнено \eqref{th861:eq1}, то в силу
  формулы Остроградского-Гаусса: для любого $\mathbb{V} \subset \mathbb{G}$ с
  кусочно-гладкой границей $\partial \mathbb{V}$ имеем:
  \begin{gather*}
    \iint\limits_{\partial \mathbb{V}_r} \pvec a d \pvec S =
    \iiint\limits_{\mathbb{V}} \mbox{div} \ \pvec a dx dy dz = 0
  \end{gather*}
\end{proof}

\begin{example}
  Если $\pvec b$ --- дважды непрерывно дифференцируемое в области $\mathbb{G}$
  поле, то $\mbox{rot} \ \pvec b$ является соленоидальным в области
  $\mathbb{G}$ полем.
  \begin{gather*}
    \mbox{div} \ \mbox{rot} \ \pvec b = (\nabla, [\nabla, \pvec b]) = 0
  \end{gather*}
\end{example}
