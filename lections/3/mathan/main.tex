\documentclass[a4paper,12pt]{book}

\usepackage[unicode,colorlinks=true,linkcolor=blue]{hyperref}
% здесь подключении шрифтов в русскими буквами
\usepackage[T2A]{fontenc}
\usepackage[utf8]{inputenc}
\usepackage[english,russian]{babel}
\usepackage{amsmath,amsthm,amssymb,amsfonts,mathtext,cite,enumerate,float}
%\usepackage[dvips]{graphicx}
\usepackage[pdftex]{graphicx}
\graphicspath{{images/}}
\usepackage{fix-cm}
\usepackage{mathenv}

\makeatletter
\renewcommand{\@biblabel}[1]{#1.}
\makeatother

\usepackage{geometry}  % Меняем поля страницы
\geometry{left=2cm}    % левое поле
\geometry{right=1.5cm} % правое поле
\geometry{top=1cm}     % верхнее поле
\geometry{bottom=2cm}  % нижнее поле

\newcommand{\HRule}{\rule{\linewidth}{0.5mm}}

\renewcommand{\theenumi}{\arabic{enumi}}                                       % Меняем везде перечисления на цифра.цифра
\renewcommand{\labelenumi}{\arabic{enumi}.}                                     % Меняем везде перечисления на цифра.цифра
\renewcommand{\theenumii}{.\arabic{enumii}}                                    % Меняем везде перечисления на цифра.цифра
\renewcommand{\labelenumii}{\arabic{enumi}.\arabic{enumii}.}                   % Меняем везде перечисления на цифра.цифра
\renewcommand{\theenumiii}{.\arabic{enumiii}}                                  % Меняем везде перечисления на цифра.цифра
\renewcommand{\labelenumiii}{\arabic{enumi}.\arabic{enumii}.\arabic{enumiii}.} % Меняем везде перечисления на цифра.цифра

\theoremstyle{plain}
\newtheorem{theorem}{Теорема}[section]
\newtheorem*{consequence}{Следствие}
\newtheorem{lemma}{Лемма}[section]
\theoremstyle{definition}
\newtheorem{definition}{Определение}[section]
\theoremstyle{remark}
\newtheorem*{comment}{Комментарий}
\newtheorem{example}{Пример}[section]
\newtheorem{exercise}{Упражнение}[]
\newtheorem{remark}{Замечание}[section]
\newtheorem{approval}{Утверждение}[section]

\begin{document}
  \begin{titlepage}
  \begin{center}

    % Upper part of the page. The '~' is needed because \\
    % only works if a paragraph has started.
    %\includegraphics[width=0.15\textwidth]{./logo}~\\[1cm]

    \textsc{\LARGE НИЯУ МИФИ}\\[1.5cm]

    \textsc{\Large Лекции 3 семестр факультет КиБ}\\[0.5cm]

    % Title
    \HRule \\[0.4cm]
    {\huge \bfseries Математический анализ\\[0.4cm]}

    \HRule \\[1.5cm]

    \begin{minipage}{0.4\textwidth}
      \begin{flushleft} \large
        \emph{Автор:}\\
        Тропин \textsc{А.Г.}
      \end{flushleft}
    \end{minipage}
    \begin{minipage}{0.4\textwidth}
      \begin{flushright} \large
        \emph{Лектор:} \\
        Теляковский \textsc{Д.С.}
      \end{flushright}
    \end{minipage}

    \vfill
    \begin{flushleft}
      e-mail: \href{mailto:andrewtropin@gmail.com}{andrewtropin@gmail.com} \\
      github: \href{http://github.com/abcdw/mephi}{abcdw/mephi}
    \end{flushleft}
    {\large \today}
  \end{center}
\end{titlepage}

  \tableofcontents

  \part{Функциональный последовательности и ряды}
  \chapter{Числовые ряды}
  \chapter{Числовые ряды}
\section{Определение}

\begin{definition}
  $U_1 + U_2 + U_3 + \dots = \sum\limits_{k = 0}^{\infty}{U_k}$
\end{definition}

\begin{definition}
  $S_n = $ - Частичная сумма
\end{definition}
\begin{definition}
  Ряд сходится, если $\exists \lim_{n \to \infty} \sum\limits_0^\infty U_k$
\end{definition}

\begin{definition}
  $\{a_n\} a_n = a_0 + \sum\limits_1^n (a_k - a_{k-1}$
\end{definition}

\begin{theorem}[Критерий Коши]
  Ряд сходится $ \Leftrightarrow \forall \varepsilon \exists N = N(\varepsilon)
  \forall n \geq N, \forall p |\sum\limits_{k=n+1}^{n+p} n_k| = |U_{n+1} + \dots
  + U_{n+p} = |S_{n+p} + S_{n}| < \varepsilon$
\end{theorem}

\begin{proof}
  $\forall \varepsilon > 0 \exists N: \forall n \geq N, \forall p
  |S_{n+p} + S_{n}| < \varepsilon$
\end{proof}

\begin{definition}
  Краевые условия
  Если ряд
\end{definition}

\begin{example}
  $\sum\limits_0^\infty z^n$
  $S_n(z) = \sum\limits_0^n z_n = \frac{1-z^{n+1}}{1-z}$
  При
\end{example}

\section{Действия с рядами}
\begin{theorem}
  Ряды $\sum U_k$ и $\sum V_k$ сходятся, тогда
  $\sum\alpha U_k = \alpha\sum U_k$\\
  $\sum U_k \pm V_k = \sum U_k \pm \sum V_k$
\end{theorem}
\begin{proof}
  $\sum\limits_{k=0}^\infty \alpha U_k = \lim_{n \to \infty}
  \sum\limits_0^n \alpha U_k = \alpha \lim_{n \to \infty} \sum\limits_0^n U_k
  = \alpha \sum\limits_0^\infty U_k$
\end{proof}

\begin{proof}
  Аналогично второе.
\end{proof}

\begin{remark}
  Сумма сходится $\not \rightarrow$ по отдельности. \\
  Еще свойство
  Нельзя раскрывать скобки и переставлять.
\end{remark}

\section{Ряды с неотрицательными членами}

$S_n$ - не строго возрастающая
Сходимость ряда эквивалентна ограниченности $S_n$

\begin{theorem}
  \begin{enumerate}
    \item $U_k \geq 0, V_k \geq 0 \forall k$
      Если $0 \leq U_k \leq V_k$, то $\sum V_k$ сходится $\Rightarrow$
      $\sum U_k$ сходится
      $\sum U_k$ расходится $\Rightarrow$ расходится $\sum V_k$ \\
    \item Если $\lim_{n\to \infty} \frac{U_k}{V_k} = A > 0$, то ряды
      сходятся или расходятся.
  \end{enumerate}
\end{theorem}

\begin{proof}
  Тут доказательство
\end{proof}

\begin{remark}
  Вместо существования предела достаточно предположить, что существуют такие числа
  p и q > 0 такие что $0 < q < \frac{U_k}{V_k} < p \forall k$
\end{remark}

\begin{theorem}[Признак Даламбера]
  Признаки\\
  \begin{enumerate}
    \item Если $\exists q \forall k \frac{U_{k+1}}{U_k} < q < 1$ сходится
    \item Если предел
  \end{enumerate}
\end{theorem}

\begin{proof}
  \begin{enumerate}
    \item Идея докозательства - сравнение с геометрической прогрессией.
    \item Для предельного случая
  \end{enumerate}
\end{proof}

\begin{theorem}[Признак Коши]
  $\sum U_k, U_k \geq 0$ \\
  \begin{enumerate}
    \item Если $\exists q < 1$, то $\forall k \sqrt[k]U_k \leq q < 1$
    \item Если $\exists \lim_{k \to \infty} \sqrt[k]U_k = q(\geq 0)$
  \end{enumerate}
  $q < 1$ - сходится \\
  $q > 1$ - расходится \\
  $q = 1$ - нужны дополнительные исследования\\
\end{theorem}
\begin{remark}
  $\overline \lim$ вместо $\lim$
\end{remark}

\begin{proof}
  Сравнение с геометрической прогрессией \\
  Если $\forall k \sqrt[k]U_k \leq q < 1 \Leftrightarrow U_k \leq q^k$
\end{proof}

\begin{definition}
  $\{a_n\}$ \\
  $\overline \lim_{n\to \infty} a_n$
\end{definition}

\begin{proof}[Признак Коши с верзним пределом]
  $\overline \lim_{k \to \infty} \sqrt[k]U_k = q < 1$
\end{proof}

\begin{remark}
  Признак Даламбера слабее признака Коши
\end{remark}

  \chapter{Функциональные последовательности и ряды}
  \section{Поточечная сходимость}
Пусть на некотором множестве $\mathbb{E}$ задана последовательность
комплексно значимых функций $f_n, n = 1, 2, \dots \ , \ (f_n \in \mathbb{C})$.
Элементы $x \in \mathbb{E}$ будем называть точками.
\begin{definition}
  $\{f_n\}$ называется ограниченной на $\mathbb{E}$, если
  $\exists M > 0: \forall n \in \mathbb{N}, \forall x \in \mathbb{E}$ выполняется
  $$|f_n(x)| \leq M$$
\end{definition}

\begin{definition}
  $\{f_n\}$ называется сходящейся поточечно на множестве $\mathbb{E}$, если
  при любом фиксированном $x \in \mathbb{E}$, числовая последовательность
  $\{f_n(x)\}$ сходится. Если последовательность сходится на $\mathbb{E}$,
  то $f(x) := \lim\limits_{n \to \infty} f_n(x), x \in \mathbb{E}$ называется
  пределом последовательности. Пусть
  $\{U_n(x)\}_{n = 1}^\infty, x \in \mathbb{E}, \ (U_n \in \mathbb{C})$
  --- последовательность числовых функций.
\end{definition}

\begin{definition}
  Множество числовых рядов
  \begin{gather}
    \sum\limits_{n = 1}^{\infty} U_n(x) \label{def213:series1}
  \end{gather}
  в каждой из которых точка $x$ фиксированная называется рядом на множестве
  $\mathbb{E}$, а функция $U_n(x)$ --- его член. \\
  $S_n(x) = \sum\limits_{k = 1}^{n} U_k(x), x \in \mathbb{E}$ называется
  $n$-ой частичной суммой ряда \ref{def213:series1}. \\
  $\sum\limits_{k = n + 1}^{\infty} U_k(x)$ - его $n$-ым остатком.
\end{definition}

\begin{definition}
  Ряд \ref{def213:series1} называется сходящимся поточечно на множестве
  $\mathbb{E}$, если последовательность $\{S_n(x)\}$ сходится поточечно на
  $\mathbb{E}$. При этом $\lim\limits_{n \to \infty} S_n(x) = S(x), x \in \mathbb{E}$
  называется суммой ряда \ref{def213:series1}.
  $$S(x) = \sum\limits_{n = 1}^{\infty} U_n(x).$$
\end{definition}

\begin{definition}
  Если ряд \ref{def213:series1} при любом $x \in \mathbb{E}$ сходится абсолютно,
  то он называется абсолютно сходящимся на множестве $\mathbb{E}$.
\end{definition}

\begin{remark}
  Беззаботная перестановка членов ряда может привести к ошибке.
\end{remark}

\section{Равномерная сходимость}
\begin{definition}
  \label{def221}
  Говорят, что функциональная последовательность $\{f_n\}_{n=1}^\infty$
  сходится равномерно на $\mathbb{E}$, если $\forall \varepsilon > 0 \
  \exists N \in \mathbb{N}: \forall n > N, \forall x \in \mathbb{E}$
  имеем
  $$|f_n(x) - f(x)| < \varepsilon$$
  Ясно, что каждая равномерно сходящаяся последовательность, сходится поточечно.
\end{definition}

\begin{comment}
  Обозначение равномерной сходимости:
  $f_n \stackrel{\mathrm{\mathbb{E}}}{\rightrightarrows} f$
\end{comment}

\begin{theorem}[Критерий Коши равномерной сходимости последовательностей]
  \label{th221}
  Для того, чтобы $\{f_n\}$ равномерно сходилась на
  $\mathbb{E} \Longleftrightarrow$
  $\forall \varepsilon > 0 \ \exists N: n, m > N, \forall x \in \mathbb{E}:$
  \begin{gather}
    |f_n(x) - f_m(x)| < \varepsilon \label{th221:uneq1}
  \end{gather}
\end{theorem}

\begin{proof}
  \hfill
  \begin{itemize}
    \item Необходимость: \\
      $f_n \stackrel{\mathrm{\mathbb{E}}}{\rightrightarrows} f$, тогда
      $\forall \varepsilon > 0, \ \exists N \in \mathbb{N}: \forall n > N,
      \forall x \in \mathbb{E} \ |f_n(x) - f(x)| < \frac{\varepsilon}{2}$. \\
      $|f_n(x) - f_m(x)| \leq |f_n(x) - f(x)| + |f(x) - f_m(x)| <
      \frac{\varepsilon}{2} + \frac{\varepsilon}{2} = \varepsilon$,
      $(\forall n, m > N, \forall x \in \mathbb{E}).$
    \item Достаточность: \\
      Пусть выполняется условие Коши, тогда $\{f_n(x)\}$, удовлетворяет
      критерию Коши сходимости числовых последовательностей и следовательно
      сходящегося числового предела, который обозначим $f(x)$. \\
      Тогда перейдя к пределу при $m \to \infty$ получим $\forall n > N,
      \forall x \in \mathbb{E}: |f_n(x) - f(x)| < \varepsilon$. \\
  \end{itemize}
\end{proof}
Иногда полезен критерий, следующий из определения \ref{def221}

\begin{theorem}
  Пусть $\lim\limits_{n \to \infty} f_n(x) = f(x), \forall x \in \mathbb{E}$. \\
  Положим $r_n = \sup|f_n(x) - f(x)|, x \in \mathbb{E}$ --- равномерное уклонение. \\
  Тогда $f_n \stackrel{\mathrm{\mathbb{E}}}{\rightrightarrows} f
  \Longleftrightarrow r_n \to 0, \ n \to \infty$. (Переформулировка определения).
\end{theorem}

\begin{proof}
  Без доказательства. \\
\end{proof}

\begin{example}
  $f_n(x) = x^n, \mathbb{E} = [0, 1)$ \\
  $\lim\limits_{n \to \infty} f_n(x) = 0, \forall \in \mathbb{E},
  r_n = \sup\limits_{x \in [0, 1)} |x^n - 0| = 1 \not \to 0, n \to \infty.$ \\
  $\{x^n\}$ не является равномерно сходящейся на $\mathbb{E}$.
\end{example}

\begin{example}
  $f_n(x) = x^n - x^{n+1}, \mathbb{E} = [0, 1]$. \\
  $f_n(x) \to 0, \forall x \in \mathbb{E}, \  f_n'(x) = nx^{n-1} - (n + 1)x^n = 0$. \\
  $x_n = \frac{n}{n+1}, \ f_n(x_n) = x_n^n(1 - x_n) < \frac{1}{n + 1}$. \\
  $r_n < \frac{1}{n + 1}$. \\
\end{example}

\begin{definition}
  \begin{gather}
    \sum\limits_{n = 1}^{\infty} U_n(x), \ x \in \mathbb{E} \label{def222:series1}
  \end{gather}
  называется равномерно сходящейся, если на множестве $\mathbb{E}$ равномерно
  сходится последовательность частичных сумм. \\
\end{definition}

Пусть $S_k(x)$ --- частичные $k$-ые суммы ряда \ref{def222:series1},
$$m \geq n: U_n(x) + \dots + U_m(x) = S_m(x) - S_n(x)$$
тогда из теоремы \ref{th221} (критерий Коши равномерной сходимости последовательности)
$\Rightarrow$ Теорема \ref{th223} (критерий Коши равномерной сходимости ряда).

\begin{theorem}[Критерий Коши равномерной сходимости ряда]
  \label{th223}
  Для того, чтобы ряд \ref{def222:series1} равномерно сходился на множестве
  $\mathbb{E} \Longleftrightarrow \forall \varepsilon > 0 \ \exists N \in \mathbb{N},
  \forall x \in \mathbb{E}: $
  \begin{gather}
    |U_n(x) + \dots + U_m(x)| < \varepsilon \label{th223:uneq1}
  \end{gather}
\end{theorem}

\begin{proof}
  Без доказательства.
\end{proof}

\begin{consequence}[Необходимый признак равномерной сходимости]
  У равномерно сходящегося ряда общий член равномерно стремится к нулю.
\end{consequence}

\begin{theorem}[Признак Вейерштрасса]
  Пусть $\{U_n\}$ --- последовательнсоть функций, определенных на $\mathbb{E}$
  и пусть $|U_n(x)| \leq a_n, \forall x \in \mathbb{E}, \forall n \in \mathbb{N}.$
  Тогда если $\sum a_n < \infty$ сходится, то следовательно $\sum U_n(x)$ сходится
  равномерно на $\mathbb{E}$.
\end{theorem}
\begin{proof}
  Если $\sum a_n$ сходится, то $\forall \varepsilon > 0 |
  \sum\limits_{k = n}^{m} U_k(x)| \leq
  \sum\limits_{k = n}^{m} a_k < \varepsilon$, при любом $x \in \mathbb{E}$,
  если только $m$ и $n$ достаточно велики, теорема \ref{th111} (критерий Коши
  сходимости числового ряда). Равномерная сходимость нашего ряда вытекает из
  теоремы \ref{th223}.
\end{proof}
\begin{remark}
  $\sum a_n$ называется мажорирующим рядом $\sum U_n(x)$.
\end{remark}

\begin{remark}
  ПРОВЕРИТЬ!!! \\
  Условие признака Вейерштрасса не являются необходимыми для равномерной
  сходимости ряда.
\end{remark}

\section{Признаки равномерной сходимости рядов Дирихле и Абеля}
\begin{theorem}
  Пусть дан ряд
  \begin{gather}
    \sum\limits_{n = 1}^{\infty} a_n(x) b_n(x), \ x \in \mathbb{E} \label{th231:series1}
  \end{gather}
  такой что:
  \begin{enumerate}
    \item $a_n(x) \in \mathbb{R}, \ b_n(x) \in \mathbb{C}, \ n = 1, 2, \dots$
    \item $a_n(x) \stackrel{\mathrm{\mathbb{E}}}{\rightrightarrows} 0$
      (Равномерная сходимость к нулю), $\{a_n(x)\}$ - монотонна.
    \item $\{b_n(x)\}, \ \sum b_n(x)$ ограничена на множестве $\mathbb{E}$.
  \end{enumerate}
  Тогда ряд \ref{th231:series1} равномерно сходится на множестве $\mathbb{E}$.
\end{theorem}

\begin{proof}
  В силу условия 3, $\exists B > 0: |B_n(x)| \leq B, \ \forall x \in \mathbb{E}, \
  \forall n \in \mathbb{N}$. \\
  $\forall x \in \mathbb{E}, m \geq n \geq 2: |b_n(x) + \dots + b_m(x)| =
  |B_m(x) - B_{n-1}(x)| \leq 2B$. \\
  $\forall \varepsilon > 0$ из условия 2 $\Rightarrow \exists N = N(\varepsilon):
  n > N(\varepsilon), \forall \in \mathbb{E}$ выполняется неравенство:
  $$0 \leq |a_n(x)| < \frac{\varepsilon}{6B}.$$
  Примениев лемму Абеля \ref{th151:cons}, получим:
  \begin{gather*}
    |a_n(x) b_n(x) + \dots + a_m(x) b_m(x)| \leq 2B \\
    (|a_n(x) + 2a_m(x)| < \varepsilon, \forall x \in \mathbb{E},
    m \geq n \geq N(\varepsilon))
  \end{gather*}
  В силу критерия Коши \ref{th223}, ряд \ref{th231:series1} сходится равномерно.
\end{proof}


  \chapter{Степенные ряды}
  \section{Радиус сходимости и круг сходимости}
\begin{definition}
  Степенной ряд --- ряд вида
  \begin{gather}
    \sum\limits_{n = 0}^{\infty} a_n(z - z_0)^n, \ z, z_0 \in \mathbb{C}, n = 0, 1, \dots
    \label{def311:series1}
  \end{gather}
  $a_n$ --- коэффициенты ряда. \\
  $\xi = z - z_0, $ тогда
  $\sum\limits_{n = 0}^{\infty} a_n \xi$,
  \begin{gather}
    \sum\limits_{n = 0}^{\infty} a_n z^n \label{def311:series2}
  \end{gather}
\end{definition}

\begin{theorem}
  \label{th311}
  Степенной ряд \eqref{def311:series2}, $\alpha = \overline{\lim} \sqrt[n]{|a_n|}$,
  \begin{gather}
    R = \frac{1}{\alpha} \label{def311:eq1}
  \end{gather}
  ($\alpha = 0 \Longleftrightarrow R = \infty, \
  \alpha = +\infty \Longleftrightarrow, R = 0$), тогда ряд \eqref{def311:series2}
  абсолютно сходится, если $|z| < R,$ и рассходится, если $|z| > R$.
\end{theorem}

\begin{proof}
  Положим $C_n = a_n z^n$. По критерию Коши заключаем, что сумма
  $\sum C_n$ сходится при $\overline{\lim\limits_{n \to \infty}} \sqrt[n]{|a_n|}
  = |z|\cdot \overline{\lim\limits_{n \to \infty}} \sqrt[n]{|a_n|} =
  \frac{|z|}{R} < 1,$ то есть $|z| < R;$ и рассходится, если $|z| > R$.
\end{proof}

\begin{definition}
  Число R называется радиусом сходимости ряда \eqref{def311:series2}. \\
  $|z| < R, z \in \mathbb{C}$ называется кругом сходимости ряда
  \eqref{def311:series2}.
\end{definition}

\begin{remark}
  О сходимсоти на границе окружности $|z| = R$ ничего не говорится в теореме
  \eqref{th311}, так как возможны все варианты.
\end{remark}

\begin{theorem}
  \label{th312}
  Если R --- радиус сходимости $(R > 0)$ ряда \eqref{def311:series2}, то на любом
  круге $|z| < r, $ где $r$ --- фиксированно, и $r < R$. \\
  этот ряд сходится абсолютно и равномерно.
\end{theorem}

\begin{proof}
  $z = r, \sum\limits_{n = 0}^{\infty} |a_n| r^n$ сходится, а так как для любой
  точки $z$ круга $|z| \leq r$ выполняется неравенство:
  \begin{gather*}
    |a_n z^n| \leq |a_n| r^n, \ \forall n
  \end{gather*}
  то по признаку Вейерштрассе на этом круге ряд \eqref{def311:series2} сходится
  равномерно.
\end{proof}

\begin{consequence}
  Степеной ряд непрерывный в каждой точке своего круга $|z| < R$ сходится.
\end{consequence}

\begin{theorem}[2-ая т. Абеля]
  \label{th313}
  Если R --- радиус сходимости, $\sum\limits_{n = 0}^{\infty} a_n z^n$ и этот
  ряд сходится при $|z| = R,$ то он сходится на отрезке $[0, R]$ равномерно.
\end{theorem}

\begin{proof}
  Пусть $0 \leq x \leq R$, представим ряд $\sum\limits_{n = 0}^{\infty} a_n x^n
  = \sum\limits_{n = 0}^{\infty} a_n R^n\left(\frac{x}{R}\right)^n$. По скольку
  члены ряда $\sum a_n R^n$ не зависит от $x$, то его сходимость означает его
  равномерную сходимость. $\{(\frac{x}{R})^n\}$ ограничена на отрезке $[0, R]$
  и монотонна в каждой точке. \\
  Поэтому в силу признака Абеля равномерной сходимости рядов \eqref{th232} ряд
  \eqref{def311:series2} равномерно сходится на отрезке $[0, R]$.
\end{proof}

\begin{lemma}
  \label{ch3:lemma1}
  Радиусы сходимости $R, R_1, R_2$ соответственно рядов
  $\sum\limits_{n = 0}^{\infty} a_n z^n, \sum\limits_{n = 0}^{\infty}
  \frac{a_n}{n + 1} z^{n+1}, \sum\limits_{n = 0}^{\infty} n a_n z^{n - 1}$ равны:
  $R = R_1 = R_2$.
\end{lemma}

\begin{proof}
  Действительно, так как $\lim\limits_{n \to \infty} \sqrt[n]{\frac{1}{n + 1}}=
  \lim\limits_{n \to \infty} \sqrt[n]{n} = 1$, то \\
  $\overline{\lim\limits_{n \to \infty}} \sqrt[n]{|a_n|} =
  \overline{\lim\limits_{n \to \infty}} \sqrt[n]{\frac{a_n}{n + 1}}=
  \overline{\lim\limits_{n \to \infty}} \sqrt[n]{|n a_n|}$
\end{proof}

\begin{example}
  $\sum a_n (z - z_0)^n$.
  Областью сходимости такого ряда является круг $|z - z_0| < R$, с точностью до
  граничных точек.
\end{example}

\section{Степенные ряды в действительной области. Общие свойства.}
В параграфах 3.2 - 3.4 будем рассматривать
\begin{gather}
  \sum\limits_{n = 0}^{\infty} a_n (x - x_0)^n,
  \label{ch3:lim1}
\end{gather}
где $a_n, x, x_0$ --- действительные числа. \\
Если $R$ --- радиус сходимости ряда ряда \eqref{ch3:lim1}, то очевидно ряд
\eqref{ch3:lim1} сходится, если $|x| < R$ и расходится, если $|x| > R$. \\
Число $R$ --- по-прежнему называется радиусом сходимости ряда \eqref{ch3:lim1},
а интервал \\ $(x_0 - R, x_0 + R)$ --- его интервал сходимости.

\begin{theorem}
  \label{th321}
  Если $R$ --- радиус сходимости ряда
  \begin{gather}
    f(x) = \sum\limits_{n = 0}^{\infty} a_n (x - x_0)^n,
    \label{ch3:lim2}
  \end{gather}
  где $R > 0$, то:
  \begin{enumerate}
    \item функция $f$ имеет в интервале $(x_0 - R, x_0 + R)$ производные всех
      порядков, они называются почленным диффиринциалом ряда \eqref{ch3:lim2}:
      \begin{gather}
        f^{(m)}(x) = \sum\limits_{n = m}^{\infty} n(n-1)\dots(n - m + 1)
        a_n(x - x_0)^{n-m}, \ m = 1, 2, \dots
        \label{ch3:lim3}
      \end{gather}
    \item $\forall x \in (x_0 - R, x_0 + R)$
      \begin{gather}
        \int\limits_{x_0}^x f(t) dt = \sum\limits_{n = 0}^{\infty} a_n
        \frac{(x - x_0)^{n+1}}{n+1}
        \label{ch3:lim4}
      \end{gather}
    \item \eqref{ch3:lim2} - \eqref{ch3:lim4} имеют одинаковые радиусы сходимости
      $R$.
  \end{enumerate}
\end{theorem}

\begin{proof}
  В силу леммы \eqref{ch3:lemma1} ряды \eqref{ch3:lim3}, \eqref{ch3:lim4} имеют
  тот же радиус сходимости, что и ряд \eqref{ch3:lim2}. Всякий ряд с $R >
  0$ сходится на отрезке $[x_0 - r, x_0 + r]$, \\
  $0 < r < R \ $(теорема~\eqref{th312}). \\
  Поэтому утверждения 1 и 2 непосредственно следуют из общих теорем о
  сходимости рядов (\eqref{th151:cons} и \eqref{th162}).
\end{proof}

\begin{theorem}
  Если функция $f$ раскладывается в некоторой окружности $x_0$, то она
  раскладывается в степенной ряд.
  \begin{gather*}
    f(x) = \sum\limits_{n = 0}^{\infty} a_n (x - x_0)^n
  \end{gather*}
  \begin{gather}
    a_n = \frac{f^{(n)}(x_0)}{n!}, \ n = 0, 1, \dots \label{ch3:lim5} \\
    f(x) = \sum\limits_{n = 0}^{\infty} \frac{f^{(n)}(x_0)}{n!} (x - x_0)^n
    \label{ch3:lim6}.
  \end{gather}
\end{theorem}

\begin{consequence}
  \label{th322:cons}
  Если в некоторой окружности точки функция раскладывается в степенной ряд, то
  это разложение единственно.
\end{consequence}

\begin{proof}
  Продифференцировав $m$ раз равенство \eqref{ch3:lim2}, получим (в силу \eqref{ch3:lim3}):

  \begin{gather*}
    f^{(m)}(x) = m(m-1)\dots 2 \cdot 1 \cdot a_m +
    (m+1)m \dots a_{m-1} (x-x_0) + (m+2)(m+1) \dots 3 \cdot a_{m-2} (x -
    x_0)^2\dots
  \end{gather*}
  Положим $x = x_0$, тогда получаем:
  \begin{gather*}
    f^{(m)}(x_0) = m! \ a_m, \ m = 0, 1, \dots
  \end{gather*}
\end{proof}

\section{Ряд Тейлора. Разложение функции в степенные ряды.}
\begin{definition}
  Пусть $f$ определена в некоторой окрестности точки $x_0$ и имеет в этой точке
  производные всех порядков, тогда ряд
  \begin{gather}
    \sum\limits_{n = 0}^{\infty} \frac{f^{(n)}(x_0)}{n!} (x-x_0)^n
    \label{def331:series1}
  \end{gather}
  Называется рядом Тейлора функции $f$ в точке $x_0$.
\end{definition}

Следующий пример показывает, что функция, бесконечно дифференцируемая в одной
точке может быть не равна разложению по Тейлору в окрестности этой точки.

\begin{example}
  \begin{gather*}
    f(x) =
    \begin{cases}
      e^{-1/x^2}, \ x \not = 0 \\
      0, \ x = 0
    \end{cases}
  \end{gather*}
  $f^{(n)}(0) = 0, \ n = 0, 1, \dots$ \\
  Отсюда следует, что все члены ряда Тейлора \eqref{ch3:lim2} в точке $x_0 = 0$,
  и не совпадают с функцией $f(x)$ в никакой окрестности точки $x_0$.
\end{example}

\begin{approval}
  \label{app331}
  Пусть функция $f(x)$ определена в некоторой окрестности $(x_0 - h, x_0 + h)$.
  \begin{gather}
    S_n(x) = \sum\limits_{k = 0}^{n} \frac{f^{(k)}(x_0)}{k!} (x - x_0)^k
    \label{app331:sum1} \\
    r_n(x) = f(x) - S_n(x) \label{app331:term1}
  \end{gather}
  Тогда, для того, чтобы функция $f(x)$ на интервале $(x_0 - h, x_0 + h)$ равна
  сумме своего ряда~\eqref{def311:series1}, то есть:
  \begin{gather}
    (S_n(x) \to f(x), \ n \to \infty) \Longleftrightarrow \lim\limits_{n \to
    \infty} r_n(x) = 0, \ \forall x \in (x_0 - h, x_0 + h)
    \label{app331:lim1}
  \end{gather}
\end{approval}

\begin{theorem}
  \label{th331}
  Пусть функция $f$ и все ее производные ограничены в совокупности на интервале
  $(x_0 - h, x_0 + h)$, то есть существует такая $M = const, M > 0$: \\$\forall
  x \in (x_0 - h, x_0 + h), \ n = 0, 1, \dots, $ выполняется неравенство:
  \begin{gather}
    |f^{(n)}(x)| \leq M \label{th331:uneq1}
  \end{gather}
  Тогда на интервале $(x_0 - h, x_0 + h)$ функция $f$ раскладывается в ряд
  Тейлора:
  \begin{gather}
    f(x) = \sum\limits_{n = 0}^{\infty} \frac{f^{(n)}(x_0)}{n!} (x - x_0)^n,
    \label{th331:series1}
  \end{gather}
  где $|x - x_0| < h$.
\end{theorem}

\begin{proof}
  \begin{gather}
    \forall a : \lim\limits_{n \to \infty} \frac{a^n}{n!} = 0 \label{th331:lim1}
  \end{gather}
  По формуле Тейлора с остаточным членом в форме Лагранжа, для $x \in (x_0 - h,
  x_0 + h)$, для $\forall M$ имеем:
  \begin{gather*}
    f(x) = S_n(x) + r_n(x),
  \end{gather*}
  где $r_n(x) = \frac{f^{(n+1)}(\xi)}{(n+1)!}(x - x_0)^{n+1}$, где $\xi = x_0 +
  \theta(x - x_0)$, где $0 < \theta < 1$. \\
  Используя \eqref{th331:uneq1} получим:
  \begin{gather*}
    |r_n(x)| = \frac{|f^{(n+1)}(\xi) (x - x_0)^{n+1}|}{(n+1)!} \leq
    \frac{M|x-x_0|^{n+1}}{(n+1)!}, \ \forall x \in (x_0 - h, x_0 + h).
  \end{gather*}
  Остюда из \eqref{th331:lim1} следует \eqref{app331:lim1}. Согласно
  утверждению \eqref{app331} теорема доказана.
\end{proof}

\section{Разложение основных элементарных в ряд Тейлора.}
\begin{itemize}
  \item Разложение в ряд функции $e^x, \cos x, \sin x$. \\
    Использую теорему \eqref{th331}, получаем:
    \begin{gather*}
      f^{(n)}(x) = e^x, \ \sin(x + \frac{\pi}{2}n), \ \cos(x + \frac{\pi}{2}n),
      \ n = 0, 1, \dots,
    \end{gather*}
    Так что $|f^{(n)}(x)| \leq e^h, \ f(x) = e^x, |x| \leq h$ \\
    $|f^{(n)}(x)| \leq 1, \ f(x) = \sin x, \cos x, \forall x \in \mathbb{R}$ \\
    Так как коэффициенты Тейлора для этих функций известны, то мы можем
    записать разложение при любом $x$:
    \begin{gather}
      e^x = \sum\limits_{n = 0}^{\infty} \frac{x^n}{n!} \label{ch34:series1} \\
      \sin x = \sum\limits_{n = 0}^{\infty} (-1)^n \frac{x^{2n+1}}{(2n+1)!}
      \label{ch34:series2} \\
      \cos x = \sum\limits_{n = 0}^{\infty} (-1)^n \frac{x^{2n}}{(2n)!}
      \label{ch34:series3}
    \end{gather}
  \item Разложение в ряд функции $\sh x, \ch x$. \\
    Заменив в \eqref{ch34:series1} $x$ на $-x$ получим
    \begin{gather}
      e^{-x} = \sum\limits_{n = 0}^{\infty} \frac{(-1)^n x^n}{n!}
      \label{ch34:series4}
    \end{gather}
    Отсюда из \eqref{ch34:series1} получаем:
    \begin{gather}
      \sh x = \frac{1}{2} \left(e^x - e^{-x}\right) = \sum\limits_{n =
      0}^{\infty} \frac{x^{2n+1}}{(2n + 1)!} \label{ch34:series5} \\
      \ch x = \frac{1}{2} \left(e^x + e^{-x}\right) = \sum\limits_{n =
      0}^{\infty} \frac{x^{2n}}{(2n)!} \label{ch34:series6}
    \end{gather}
    В правых частях этих формул разложения степенных функций в ряды единственно
    в силу теоремы \eqref{th322:cons}.
  \item Разложение в ряд функции $\ln(1 + x)$. \\
    Рассмотрим:
    \begin{gather}
      \frac{1}{1 + t} = 1 - t + t^2 - t^3 + \dots + (-1)^n t^n + \dots, \ |t| < 1
      \label{ch34:series7}
    \end{gather}
    Интегрирую его почленно по теореме \eqref{th321} от $0$ до $x \in (-1, 1)$
    получим:
    \begin{gather*}
      \int\limits_0^x \frac{dt}{1+t} = \ln(1+x) = x - \frac{x^2}{2} +
      \frac{x^3}{3} - \dots, \\
      \ln(1+x) = \sum\limits_{n = 1}^{\infty} \frac{(-1)^{n+1} x^n}{n}, \
      \forall x \in (-1, 1)
      \label{ch34:series8}
    \end{gather*}
    Ряд правой части равенства \eqref{ch34:series8} сходится по признаку Лейбница
    $\Rightarrow$ согласно теореме Абеля \eqref{th313}, разложение
    \eqref{ch34:series8} имеет место в промежутке $(-1, 1]$
  \item Разложение в ряд $(1 + x)^\alpha, \alpha \not = 0, 1, \dots$
    Формула Тейлора для этой функции имеет вид:
    \begin{gather}
      (1+x)^\alpha = 1 + \alpha x + \frac{\alpha(\alpha - 1)}{2!}x^2 + \dots +
      \frac{\alpha(\alpha - 1)\dots(\alpha - n + 1)}{n!} x^n + r_n(x)
      \label{ch34:series9}
    \end{gather}
    Соответствующий степенной ряд называют
    \begin{gather}
      1 + \sum\limits_{n = 1}^{\infty} \frac{\alpha(\alpha - 1)\dots(\alpha
      -n + 1)}{n!} x^n \label{ch34:series10}
    \end{gather}
    биномиальным рядом. \\
    $R = \lim\limits_{n \to \infty} \left|\frac{a_n}{a_{n+1}}\right| =
    \lim\limits_{n \to \infty} \left|\frac{n+1}{\alpha - n}\right| = 1$,
    в силу утверждения, $r_n(x) \to 0$.

    \begin{remark}
      Поведение ряда \eqref{ch34:series10} в точках $\pm 1$, характерезуется
      следующей таблицой:
      \begin{table}[H]
        \caption{таблица, характеризующая ряд \eqref{ch34:series10}}
        \begin{center}
          \begin{tabular}{|c|c|l|}
            \hline
            & $\alpha > 0$ & абсолютно сходится \\
            $x = 1$ & $-1 < \alpha < 0$ & условно сходится \\
            & $\alpha \leq -1$ & расходится\\
            \hline
            $x = -1$ & $ \alpha > 0$ & абсолютно сходится \\
            & $\alpha < 0$ & рассходится \\
            \hline
          \end{tabular}
        \end{center}
      \end{table}
      Согласно второй теореме Абеля \eqref{th313} всякий раз, когда ряд
      \eqref{ch34:series10} сходится при $x = \pm 1$, его сумма равна $(1 +
      x)^\alpha$.
    \end{remark}
  \item Разложение в ряд $\arctg x$ \\
    Рассмотрим ряд:
    \begin{gather}
      \frac{1}{1 + t^2} = 1 - t^2 + t^4 - t^6 + \dots + (-1)^n t^{2n} + \dots, \ |t| < 1
    \end{gather}
    Интегрирую его почленно по теореме \eqref{th321} от $0$ до $x \in (-1, 1)$
    получим:
    \begin{gather*}
      \int\limits_0^x \frac{dt}{1+t^2} = \arctg x = x - \frac{x^3}{3} +
      \frac{x^5}{5} - \dots + (-1)^n \frac{x^{2n+1}}{(2n+1)} + \dots
    \end{gather*}
    Ряд правой части равенства \eqref{ch34:series8} сходится по признаку Лейбница
    $\Rightarrow$ согласно теореме Абеля \eqref{th313}, разложение
    \eqref{ch34:series8} имеет место на отрезке $(-1, 1)$. \\
    В частности, при $x = 1$, получим:
    \begin{gather*}
      1 - \frac{1}{3} + \frac{1}{5} - \frac{1}{7} + \dots = \frac{\pi}{4}
    \end{gather*}
  \item Разложение в ряд $\arcsin x$ \\
    Рассмотрим ряд:
    \begin{gather}
      \frac{1}{\sqrt{1 - t^2}} = 1 + \sum\limits_{n = 1}^{\infty}
      \frac{(2n-1)!!}{(2n)!!}t^{2n}, \ |t| < 1
    \end{gather}
    Интегрирую его почленно по теореме \eqref{th321} от $0$ до $x \in (-1, 1)$
    получим:
    \begin{gather*}
      \int\limits_0^x \frac{dt}{\sqrt{1-t^2}} = \arcsin x = x + \sum\limits_{n
      = 1}^{\infty} \frac{(2n-1)!!}{(2n)!!} \frac{x^{2n+1}}{2n+1}, \ |x| \leq 1
    \end{gather*}
    Справедливость этого разложения при $x = \pm 1$ устанавливается с помощью
    второй теоремы Абеля \eqref{th313}.
\end{itemize}

\section{Формулы Эйлера}
Ряды разложения \eqref{ch34:series1} - \eqref{ch34:series3} функций $e^x, \sin x,
\cos x$ сходятся всюду в комплексной плоскости $\mathbb{C}$. По этой причине
естественны следующие определения($e^z, \sin z, \cos z, z \in \mathbb{C}$):
\begin{gather}
  e^z = \exp(z) = \sum\limits_{n = 0}^{\infty} \frac{z^n}{n!}
  \label{ch35:series1} \\
  \sin z = \sum\limits_{n = 0}^{\infty} \frac{(-1)^n z^{2n+1}}{(2n+1)!}
  \label{ch35:series2} \\
  \cos z = \sum\limits_{n = 0}^{\infty} \frac{(-1)^n z^{2n}}{(2n)!}
  \label{ch35:series3}
\end{gather}
Заменив $z$ сначала на $iz$, а затем на $-iz$ получим:
\begin{gather}\label{ch35:series4}
  e^{iz} = \sum\limits_{n = 0}^{\infty} \frac{i^n z^n}{n!} \\
  e^{-iz} = \sum\limits_{n = 0}^{\infty} \frac{(-1)^n i^n z^n}{n!}
\end{gather}

Заметим, что $i^{2k} = (-1)^k, i^{2k + 1} = (-1)^ki, k = 0, 1, \dots$. \\
\begin{gather*}
  \frac{1}{2} (e^{iz} + e^{-iz}) = \sum\limits_{n = 0}^{\infty} \frac{(-1)^n
  z^{2n}}{(2n)!} \\
  \frac{1}{2i} (e^{iz} - e^{-iz}) = \sum\limits_{n = 0}^{\infty}
  \frac{(-1)^n z^{2n+1}}{(2n+1)!}
\end{gather*}
Сравнив эти формулы с \eqref{ch35:series2}, \eqref{ch35:series3} заключаем, что
\begin{gather}
  \cos z = \frac{e^{iz} + e^{-iz}}{2}
  \label{ch35:eq1} \\
  \sin z = \frac{e^{iz} - e^{-iz}}{2i}
  \label{ch35:eq2}
\end{gather}

Из этих формул следует формула:
\begin{gather}
  \cos z + i \sin z = e^{iz} \label{ch35:eq3}
\end{gather}
Формулы \eqref{ch35:eq1}, \eqref{ch35:eq2} и \eqref{ch35:eq3} называются
формулами Эйлера. \\
Если в формуле \eqref{ch35:eq3} $z = \varphi, \ \varphi \in
\mathbb{R}$, то
\begin{gather*}
  \cos \varphi + i \sin \varphi = e^{i\varphi}
\end{gather*}
Поэтому $z \in \mathbb{C}, |z| = r, \ z = r(\cos \varphi + i \sin
\varphi)$
\begin{gather*}
  z = r e^{i\varphi}
\end{gather*}

\begin{definition}
  $W(x) = U(x) + iV(x), \ x \in \mathbb{R}, V(x) \in \mathbb{R}$ \\
  Положим $\frac{dW}{dx} = U'(x) + iV'(x)$, тогда
  \begin{gather*}
    \int\limits_a^b W(x) dx = \int\limits_a^b U(x) dx + i \int\limits_a^b V(x)
    dx
  \end{gather*}
\end{definition}
$e^{i\pi} = -1$

  %\chapter{Радомский консультация}
  %\input{arad}
  \chapter{Ряды Фурье}
  \section{Ортогональные системы} \label{ch41}
В параграфах \eqref{ch41} - \eqref{ch43} $\mathbb{X}$ --- линейное
бесконечномерное пространство(действительное или комплексное, со скалярным
произведением).
$$X(\cdot \ , \ \cdot), \ \|x\| = \sqrt{(x, x)}.$$
$\mathbb{K}$ --- некоторое счетное или конечное множество.
\begin{definition}
  \label{def411}
  Система векторов $\{x_k: k \in \mathbb{K}\}, \ x \in \mathbb{X}$ ---
  ортогональная система(ОС). \\
  $(x_i, x_j) = 0, \ \forall i, j \in \mathbb{K}, i \not = j$ (и система не
  нулевая). Если $(x_i, x_i) = 1$, то система называется ортонормированной.
\end{definition}

\begin{theorem}
  Ортогональная система векторов линейно независима, то есть линейно не
  зависима каждая ее конечная подсистема.
\end{theorem}
\begin{proof}
  Определение линейной независимости:
  \begin{gather*}
    \alpha_1 x_1 + \dots + \alpha_i x_i +
    \dots = 0 \Longleftrightarrow \alpha_i = 0, \ \forall i
  \end{gather*}
  Скалярно умножим все члены на $x_i$, тогда получим:
  \begin{gather*}
    \label{th411:eq1}
    \alpha_1 (x_1, x_i) + \dots + \alpha_i (x_i, x_i) + \dots = (0, x_i) \\
    \label{th411:eq2}
    \alpha_i (x_i, x_i) = 0 \\
    \label{th411:eq3}
    \alpha_i = 0
  \end{gather*}
  Равенство \eqref{th411:eq2} следует из определения \eqref{def411}, равенство
  \eqref{th411:eq3} следует из того, что $(x_i, x_i)~\not=~0$ (так как система
  не нулевая).
\end{proof}

\section{Коэффициенты Фурье}
\begin{definition}
  Пусть $\{e_k: \ k \in \mathbb{K}\}$ --- ОНС в $\mathbb{X}$, $\{(x, e_k)\}, x
  \in \mathbb{X}$ называется коэффициентами Фурье элемента $x$ в ОНС $e_k$.
\end{definition}
\begin{lemma}
  \label{lemma421}
  Если система векторов $e_1, \dots, e_n$ пространства $\mathbb{X}$ --- ОН, то
  $\forall x \in \mathbb{X}$ вектор $h = x - x_e$, где
  \begin{gather}
    x_e = \sum\limits_{k = 1}^{n} (x, e_k) e_k
    \label{lemma1:eq1}
  \end{gather}
  ортоганален подпространству $\mathbb{L} = \langle e_1, \dots, e_n \rangle$ (натянотому
  на векторы $e_1, \dots, e_n$)
\end{lemma}

\begin{proof}
  Достаточно проверить, что скалярное произведение \\ $(h, e_j) = 0, \ \forall j =
  1, \dots, n$
  \begin{gather*}
    (h, e_j) = (x, e_j) - \sum\limits_{k = 1}^{n} (x, e_k) (e_k, e_j) =
    (x, e_j) - (x, e_j) = 0
  \end{gather*}
\end{proof}

\begin{lemma}[теорема Пифагора]
  \label{lemma422}
  Если векторы $x_1, \dots, x_n$ попарно ортогональны и $x = x_1 + \dots +
  x_n$, то $\|x\|^2 = \|x_1\|^2 + \dots + \|x_n\|^2$
\end{lemma}

\begin{proof}
  $(x, x) = (\sum\limits_{i = 1}^{n} x_i, \sum\limits_{i = 1}^{n} x_i) =
  \sum\limits_{i, j = 1}^{n} (x_i, x_j) = \sum\limits_{i = 1}^{n} (x_i, x_i)$
\end{proof}

\begin{theorem}[экстремальное свойство коэффициентов Фурье]
  \label{th423}
  Если $e_1, \dots, e_n$ --- ОНС пространства $\mathbb{X}$, то $\forall x \in
  \mathbb{X}$ и $\forall y = \alpha_1 e_1 + \dots + \alpha_n e_n$ имеет место
  неравенство:
  \begin{gather*}
    \|x - \sum\limits_{k = 1}^{n} (x, e_k) e_k \| \leq \|x - \sum\limits_{k =
    1}^{n} \alpha_k e_k \|,
  \end{gather*}
  в котором равенство возможно при условии: $\alpha_k = (x, e_k) \ \forall k = 1,
  \dots, n$.
\end{theorem}

\begin{proof}
  Представим $x - y$ в виде $x - y = (x_e - y) + h$, где $x_e, h$ определены в
  лемме~\eqref{lemma421}. \\
  По лемме~\eqref{lemma421} $h \perp (x_e - y) \in \mathbb{L}$. По теореме
  Пифагора (лемма~\ref{lemma422}):
  \begin{gather*}
    \|x - y\|^2 = \|x_e - y\|^2 + \|h\|^2 = \|x_e - y\|^2 + \|x - x_e\|^2 \geq
    \|x - x_e\|^2
  \end{gather*}
  равенство возможно, когда коэффициенты $\alpha_k$ совпадают с коэффициентами
  Фурье.
\end{proof}

\begin{remark}
  Теорема \eqref{th423} показывает, что вектор $x_e$ является наилучшей в
  смысле нормы пространства $\mathbb{X}$, аппроксимацией вектора $x$
  подпространства $\mathbb{L} = \langle e_1, \dots, e_n \rangle$, так что
  наименьшее уклонением вектора $x$ от $\mathbb{L}$ равно $\|x - x_e\|$.
\end{remark}

\section{секция}
\label{ch43}

\end{document}

