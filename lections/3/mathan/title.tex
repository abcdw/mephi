\begin{titlepage}
  \begin{center}

    % Upper part of the page. The '~' is needed because \\
    % only works if a paragraph has started.
    %\includegraphics[width=0.15\textwidth]{./logo}~\\[1cm]

    \textsc{\LARGE НИЯУ МИФИ}\\[1.5cm]

    \textsc{\Large 3-ий семестр факультет КиБ, конспект лекций}\\[0.5cm]

    % Title
    \HRule \\[0.4cm]
    {\huge \bfseries Математический анализ\\[0.4cm]}

    \HRule \\[1.5cm]

    \begin{minipage}{0.4\textwidth}
      \begin{flushleft} \large
        \emph{Автор:}\\
        Тропин \textsc{А.Г.} \\
        \emph{Соавтор:}\\
        Коверко \textsc{Е.А.}
      \end{flushleft}
    \end{minipage}
    \begin{minipage}{0.4\textwidth}
      \begin{flushright} \large
        \emph{Лектор:} \\
        Севастьянов \textsc{Е.А.}
      \end{flushright}
    \end{minipage}

    \vfill
    \begin{flushleft}
      e-mail: \href{mailto:andrewtropin@gmail.com}{andrewtropin@gmail.com} \\
      github: \href{http://github.com/abcdw/mephi}{abcdw/mephi}
    \end{flushleft}
    {\large \today}
  \end{center}
\end{titlepage}

{\bfseries Предисловие} \\
\HRule \\[0.4cm]

Данный конспект был составлен для упрощения процесса подготовки к экзамену по
математическому анализу. Может быть Евгению Александровичу больше не
придется таскать с собой здоровенную стопку перфокарт.

Хочется выразить отдельную благодарность Коверко Егору, который предоставил
большую часть материала и помог в написании этого творения. Авторы и соавторы
не несут абсолютно никаких гарантий за правильность сего произведения.


