\section{Радиус сходимости и круг сходимости}
\begin{definition}
  Степенной ряд --- ряд вида
  \begin{gather}
    \sum\limits_{n = 0}^{\infty} a_n(z - z_0)^n, \ z, z_0 \in \mathbb{C}, n = 0, 1, \dots
    \label{def311:series1}
  \end{gather}
  $a_n$ --- коэффициенты ряда. \\
  $\xi = z - z_0, $ тогда
  $\sum\limits_{n = 0}^{\infty} a_n \xi^n$,
  \begin{gather}
    \sum\limits_{n = 0}^{\infty} a_n z^n \label{def311:series2}
  \end{gather}
\end{definition}

\begin{theorem}
  \label{th311}
  Степенной ряд \eqref{def311:series2}, $\alpha = \overline{\lim\limits_{n \to
  \infty}} \sqrt[n]{|a_n|}$,
  \begin{gather}
    R = \frac{1}{\alpha} \label{def311:eq1}
  \end{gather}
  ($\alpha = 0 \Longleftrightarrow R = \infty, \
  \alpha = +\infty \Longleftrightarrow, R = 0$), тогда ряд \eqref{def311:series2}
  абсолютно сходится, если $|z| < R,$ и рассходится, если $|z| > R$.
\end{theorem}

\begin{proof}
  Положим $C_n = a_n z^n$. По критерию Коши заключаем, что сумма
  $\sum C_n$ сходится при $\overline{\lim\limits_{n \to \infty}} \sqrt[n]{|C_n|}
  = |z|\cdot \overline{\lim\limits_{n \to \infty}} \sqrt[n]{|a_n|} =
  \frac{|z|}{R} < 1,$ то есть $|z| < R;$ и рассходится, если $|z| > R$.
\end{proof}

\begin{definition}
  Число R называется радиусом сходимости ряда \eqref{def311:series2}. \\
  $|z| < R, z \in \mathbb{C}$ называется кругом сходимости ряда
  \eqref{def311:series2}.
\end{definition}

\begin{consequence}[1-ая т. Абеля]
  Если степенной ряд сходится при $z \not = 0$, то он абсолютно сходится при
  любом $|z| < |z*|$.
\end{consequence}

\begin{remark}
  О сходимости на границе окружности $|z| = R$ ничего не говорится в теореме
  \eqref{th311}, так как возможны все варианты.
\end{remark}

\begin{theorem}
  \label{th312}
  Если R --- радиус сходимости $(R > 0)$ ряда \eqref{def311:series2}, то на любом
  круге $|z| < r, $ где $r$ --- фиксированно, и $r < R$. \\
  Таким образом этот ряд сходится абсолютно и равномерно.
\end{theorem}

\begin{proof}
  $z = r, \sum\limits_{n = 0}^{\infty} |a_n| r^n$ сходится, а так как для любой
  точки $z$ круга $|z| \leq r$ выполняется неравенство:
  \begin{gather*}
    |a_n z^n| \leq |a_n| r^n, \ \forall n
  \end{gather*}
  то по признаку Вейерштрассе на этом круге ряд \eqref{def311:series2} сходится
  равномерно.
\end{proof}

\begin{consequence}
  Степеной ряд непрерывный в каждой точке своего круга $|z| < R$ сходится.
\end{consequence}

\begin{theorem}[2-ая т. Абеля]
  \label{th313}
  Если R --- радиус сходимости, $\sum\limits_{n = 0}^{\infty} a_n z^n$ и этот
  ряд сходится при $|z| = R,$ то он сходится на отрезке $[0, R]$ равномерно.
\end{theorem}

\begin{proof}
  Пусть $0 \leq x \leq R$, представим ряд $\sum\limits_{n = 0}^{\infty} a_n x^n
  = \sum\limits_{n = 0}^{\infty} a_n R^n\left(\frac{x}{R}\right)^n$. По скольку
  члены ряда $\sum a_n R^n$ не зависит от $x$, то его сходимость означает его
  равномерную сходимость. $\{(\frac{x}{R})^n\}$ ограничена на отрезке $[0, R]$
  и монотонна в каждой точке. \\
  Поэтому в силу признака Абеля равномерной сходимости рядов \eqref{th232} ряд
  \eqref{def311:series2} равномерно сходится на отрезке $[0, R]$.
\end{proof}

\begin{lemma}
  \label{ch3:lemma1}
  Радиусы сходимости $R, R_1, R_2$ соответственно рядов
  $\sum\limits_{n = 0}^{\infty} a_n z^n, \sum\limits_{n = 0}^{\infty}
  \frac{a_n}{n + 1} z^{n+1}, \sum\limits_{n = 0}^{\infty} n a_n z^{n - 1}$ равны:
  $R = R_1 = R_2$.
\end{lemma}

\begin{proof}
  Действительно, так как $\lim\limits_{n \to \infty} \sqrt[n]{\frac{1}{n + 1}}=
  \lim\limits_{n \to \infty} \sqrt[n]{n} = 1$, то \\
  $\overline{\lim\limits_{n \to \infty}} \sqrt[n]{|a_n|} =
  \overline{\lim\limits_{n \to \infty}} \sqrt[n]{\frac{a_n}{n + 1}}=
  \overline{\lim\limits_{n \to \infty}} \sqrt[n]{|n a_n|}$
\end{proof}

\begin{example}
  $\sum a_n (z - z_0)^n$.
  Областью сходимости такого ряда является круг $|z - z_0| < R$, с точностью до
  граничных точек.
\end{example}

\section{Степенные ряды в действительной области. Общие свойства.}
В параграфах 3.2 - 3.4 будем рассматривать
\begin{gather}
  \sum\limits_{n = 0}^{\infty} a_n (x - x_0)^n,
  \label{ch3:lim1}
\end{gather}
где $a_n, x, x_0$ --- действительные числа. \\
Если $R$ --- радиус сходимости ряда ряда \eqref{ch3:lim1}, то очевидно ряд
\eqref{ch3:lim1} сходится, если $|x| < R$ и расходится, если $|x| > R$. \\
Число $R$ --- по-прежнему называется радиусом сходимости ряда \eqref{ch3:lim1},
а интервал \\ $(x_0 - R, x_0 + R)$ --- его интервал сходимости.

\begin{theorem}
  \label{th321}
  Если $R$ --- радиус сходимости ряда
  \begin{gather}
    f(x) = \sum\limits_{n = 0}^{\infty} a_n (x - x_0)^n,
    \label{ch3:lim2}
  \end{gather}
  где $R > 0$, то:
  \begin{enumerate}
    \item функция $f$ имеет в интервале $(x_0 - R, x_0 + R)$ производные всех
      порядков, они называются почленным диффиринциалом ряда \eqref{ch3:lim2}:
      \begin{gather}
        f^{(m)}(x) = \sum\limits_{n = m}^{\infty} n(n-1)\dots(n - m + 1)
        a_n(x - x_0)^{n-m}, \ m = 1, 2, \dots
        \label{ch3:lim3}
      \end{gather}
    \item $\forall x \in (x_0 - R, x_0 + R)$
      \begin{gather}
        \int\limits_{x_0}^x f(t) dt = \sum\limits_{n = 0}^{\infty} a_n
        \frac{(x - x_0)^{n+1}}{n+1}
        \label{ch3:lim4}
      \end{gather}
    \item \eqref{ch3:lim2} - \eqref{ch3:lim4} имеют одинаковые радиусы сходимости
      $R$.
  \end{enumerate}
\end{theorem}

\begin{proof}
  В силу леммы \eqref{ch3:lemma1} ряды \eqref{ch3:lim3}, \eqref{ch3:lim4} имеют
  тот же радиус сходимости, что и ряд \eqref{ch3:lim2}. Всякий ряд с $R >
  0$ сходится на отрезке $[x_0 - r, x_0 + r]$, \\
  $0 < r < R \ $(теорема~\eqref{th312}). \\
  Поэтому утверждения 1 и 2 непосредственно следуют из общих теорем о
  сходимости рядов (\eqref{th151:cons} и \eqref{th162}).
\end{proof}

\begin{theorem}
  Если функция $f$ раскладывается в некоторой окрестности точки $x_0$ в степенной ряд:
  \begin{gather*}
    f(x) = \sum\limits_{n = 0}^{\infty} a_n (x - x_0)^n
  \end{gather*},то
  \begin{gather}
    a_n = \frac{f^{(n)}(x_0)}{n!}, \ n = 0, 1, \dots \label{ch3:lim5} \\
    и следовательно справедливо:
    f(x) = \sum\limits_{n = 0}^{\infty} \frac{f^{(n)}(x_0)}{n!} (x - x_0)^n
    \label{ch3:lim6}.
  \end{gather}
\end{theorem}

\begin{consequence}
  \label{th322:cons}
  Если в некоторой окрестности точки функция раскладывается в степенной ряд, то
  это разложение единственно.
\end{consequence}

\begin{proof}
  Продифференцировав $m$ раз равенство \eqref{ch3:lim2}, получим (в силу \eqref{ch3:lim3}):

  \begin{gather*}
    f^{(m)}(x) = m(m-1)\dots 2 \cdot 1 \cdot a_m +
    (m+1)m \dots a_{m-1} (x-x_0) + (m+2)(m+1) \dots 3 \cdot a_{m-2} (x -
    x_0)^2\dots
  \end{gather*}
  Положим $x = x_0$, тогда получаем:
  \begin{gather*}
    f^{(m)}(x_0) = m! \ a_m, \ m = 0, 1, \dots
  \end{gather*}
\end{proof}

\section{Ряд Тейлора. Разложение функции в степенные ряды.}
\begin{definition}
  Пусть $f$ определена в некоторой окрестности точки $x_0$ и имеет в этой точке
  производные всех порядков, тогда ряд
  \begin{gather}
    \sum\limits_{n = 0}^{\infty} \frac{f^{(n)}(x_0)}{n!} (x-x_0)^n
    \label{def331:series1}
  \end{gather}
  Называется рядом Тейлора функции $f$ в точке $x_0$.
\end{definition}

Следующий пример показывает, что функция, бесконечно дифференцируемая в одной
точке может быть не равна разложению по Тейлору в окрестности этой точки.

\begin{example}
  \begin{gather*}
    f(x) =
    \begin{cases}
      e^{-1/x^2}, \ x \not = 0 \\
      0, \ x = 0
    \end{cases}
  \end{gather*}
  $f^{(n)}(0) = 0, \ n = 0, 1, \dots$ \\
  Отсюда следует, что все члены ряда Тейлора \eqref{ch3:lim2} в точке $x_0 = 0$,
  и не совпадают с функцией $f(x)$ в никакой окрестности точки $x_0$.
\end{example}

\begin{approval}
  \label{app331}
  Пусть функция $f(x)$ определена в некоторой окрестности $(x_0 - h, x_0 + h)$.
  \begin{gather}
    S_n(x) = \sum\limits_{k = 0}^{n} \frac{f^{(k)}(x_0)}{k!} (x - x_0)^k
    \label{app331:sum1} \\
    r_n(x) = f(x) - S_n(x) \label{app331:term1}
  \end{gather}
  Тогда, для того, чтобы функция $f(x)$ на интервале $(x_0 - h, x_0 + h)$ равна
  сумме своего ряда~\eqref{def311:series1}, то есть:
  \begin{gather}
    (S_n(x) \to f(x), \ n \to \infty) \Longleftrightarrow \lim\limits_{n \to
    \infty} r_n(x) = 0, \ \forall x \in (x_0 - h, x_0 + h)
    \label{app331:lim1}
  \end{gather}
\end{approval}

\begin{theorem}
  \label{th331}
  Пусть функция $f$ и все ее производные ограничены в совокупности на интервале
  $(x_0 - h, x_0 + h)$, то есть существует такая $M = const, M > 0$: \\$\forall
  x \in (x_0 - h, x_0 + h), \ n = 0, 1, \dots, $ выполняется неравенство:
  \begin{gather}
    |f^{(n)}(x)| \leq M \label{th331:uneq1}
  \end{gather}
  Тогда на интервале $(x_0 - h, x_0 + h)$ функция $f$ раскладывается в ряд
  Тейлора:
  \begin{gather}
    f(x) = \sum\limits_{n = 0}^{\infty} \frac{f^{(n)}(x_0)}{n!} (x - x_0)^n,
    \label{th331:series1}
  \end{gather}
  где $|x - x_0| < h$.
\end{theorem}

\begin{proof}
  \begin{gather}
    \forall a : \lim\limits_{n \to \infty} \frac{a^n}{n!} = 0 \label{th331:lim1}
  \end{gather}
  По формуле Тейлора с остаточным членом в форме Лагранжа, для $x \in (x_0 - h,
  x_0 + h)$, для $\forall M$ имеем:
  \begin{gather*}
    f(x) = S_n(x) + r_n(x),
  \end{gather*}
  где $r_n(x) = \frac{f^{(n+1)}(\xi)}{(n+1)!}(x - x_0)^{n+1}$, где $\xi = x_0 +
  \theta(x - x_0)$, где $0 < \theta < 1$. \\
  Используя \eqref{th331:uneq1} получим:
  \begin{gather*}
    |r_n(x)| = \frac{|f^{(n+1)}(\xi) (x - x_0)^{n+1}|}{(n+1)!} \leq
    \frac{M|x-x_0|^{n+1}}{(n+1)!}, \ \forall x \in (x_0 - h, x_0 + h).
  \end{gather*}
  Остюда из \eqref{th331:lim1} следует \eqref{app331:lim1}. Согласно
  утверждению \eqref{app331} теорема доказана.
\end{proof}

\section{Разложение основных элементарных в ряд Тейлора.}
\begin{itemize}
  \item Разложение в ряд функции $e^x, \cos x, \sin x$. \\
    Использую теорему \eqref{th331}, получаем:
    \begin{gather*}
      f^{(n)}(x) = e^x, \ \sin(x + \frac{\pi}{2}n), \ \cos(x + \frac{\pi}{2}n),
      \ n = 0, 1, \dots,
    \end{gather*}
    Так что $|f^{(n)}(x)| \leq e^h, \ f(x) = e^x, |x| \leq h$ \\
    $|f^{(n)}(x)| \leq 1, \ f(x) = \sin x, \cos x, \forall x \in \mathbb{R}$ \\
    Так как коэффициенты Тейлора для этих функций известны, то мы можем
    записать разложение при любом $x$:
    \begin{gather}
      e^x = \sum\limits_{n = 0}^{\infty} \frac{x^n}{n!} \label{ch34:series1} \\
      \sin x = \sum\limits_{n = 0}^{\infty} (-1)^n \frac{x^{2n+1}}{(2n+1)!}
      \label{ch34:series2} \\
      \cos x = \sum\limits_{n = 0}^{\infty} (-1)^n \frac{x^{2n}}{(2n)!}
      \label{ch34:series3}
    \end{gather}
  \item Разложение в ряд функции $\sh x, \ch x$. \\
    Заменив в \eqref{ch34:series1} $x$ на $-x$ получим
    \begin{gather}
      e^{-x} = \sum\limits_{n = 0}^{\infty} \frac{(-1)^n x^n}{n!}
      \label{ch34:series4}
    \end{gather}
    Отсюда из \eqref{ch34:series1} получаем:
    \begin{gather}
      \sh x = \frac{1}{2} \left(e^x - e^{-x}\right) = \sum\limits_{n =
      0}^{\infty} \frac{x^{2n+1}}{(2n + 1)!} \label{ch34:series5} \\
      \ch x = \frac{1}{2} \left(e^x + e^{-x}\right) = \sum\limits_{n =
      0}^{\infty} \frac{x^{2n}}{(2n)!} \label{ch34:series6}
    \end{gather}
    В правых частях этих формул разложения степенных функций в ряды единственно
    в силу теоремы \eqref{th322:cons}.
  \item Разложение в ряд функции $\ln(1 + x)$. \\
    Рассмотрим:
    \begin{gather}
      \frac{1}{1 + t} = 1 - t + t^2 - t^3 + \dots + (-1)^n t^n + \dots, \ |t| < 1
      \label{ch34:series7}
    \end{gather}
    Интегрирую его почленно по теореме \eqref{th321} от $0$ до $x \in (-1, 1)$
    получим:
    \begin{gather*}
      \int\limits_0^x \frac{dt}{1+t} = \ln(1+x) = x - \frac{x^2}{2} +
      \frac{x^3}{3} - \dots, \\
      \ln(1+x) = \sum\limits_{n = 1}^{\infty} \frac{(-1)^{n+1} x^n}{n}, \
      \forall x \in (-1, 1)
      \label{ch34:series8}
    \end{gather*}
    Ряд правой части равенства \eqref{ch34:series8} сходится по признаку Лейбница
    $\Rightarrow$ согласно теореме Абеля \eqref{th313}, разложение
    \eqref{ch34:series8} имеет место в промежутке $(-1, 1]$
  \item Разложение в ряд $(1 + x)^\alpha, \alpha \not = 0, 1, \dots$
    Формула Тейлора для этой функции имеет вид:
    \begin{gather}
      (1+x)^\alpha = 1 + \alpha x + \frac{\alpha(\alpha - 1)}{2!}x^2 + \dots +
      \frac{\alpha(\alpha - 1)\dots(\alpha - n + 1)}{n!} x^n + r_n(x)
      \label{ch34:series9}
    \end{gather}
    Соответствующий степенной ряд называют
    \begin{gather}
      1 + \sum\limits_{n = 1}^{\infty} \frac{\alpha(\alpha - 1)\dots(\alpha
      -n + 1)}{n!} x^n \label{ch34:series10}
    \end{gather}
    биномиальным рядом. \\
    $R = \lim\limits_{n \to \infty} \left|\frac{a_n}{a_{n+1}}\right| =
    \lim\limits_{n \to \infty} \left|\frac{n+1}{\alpha - n}\right| = 1$,
    в силу утверждения, $r_n(x) \to 0$.

    \begin{remark}
      Поведение ряда \eqref{ch34:series10} в точках $\pm 1$, характерезуется
      следующей таблицой:
      \begin{table}[H]
        \caption{таблица, характеризующая ряд \eqref{ch34:series10}}
        \begin{center}
          \begin{tabular}{|c|c|l|}
            \hline
            & $\alpha > 0$ & абсолютно сходится \\
            $x = 1$ & $-1 < \alpha < 0$ & условно сходится \\
            & $\alpha \leq -1$ & расходится\\
            \hline
            $x = -1$ & $ \alpha > 0$ & абсолютно сходится \\
            & $\alpha < 0$ & рассходится \\
            \hline
          \end{tabular}
        \end{center}
      \end{table}
      Согласно второй теореме Абеля \eqref{th313} всякий раз, когда ряд
      \eqref{ch34:series10} сходится при $x = \pm 1$, его сумма равна $(1 +
      x)^\alpha$.
    \end{remark}
  \item Разложение в ряд $\arctg x$ \\
    Рассмотрим ряд:
    \begin{gather}
      \frac{1}{1 + t^2} = 1 - t^2 + t^4 - t^6 + \dots + (-1)^n t^{2n} + \dots, \ |t| < 1
    \end{gather}
    Интегрирую его почленно по теореме \eqref{th321} от $0$ до $x \in (-1, 1)$
    получим:
    \begin{gather*}
      \int\limits_0^x \frac{dt}{1+t^2} = \arctg x = x - \frac{x^3}{3} +
      \frac{x^5}{5} - \dots + (-1)^n \frac{x^{2n+1}}{(2n+1)} + \dots
    \end{gather*}
    Ряд правой части равенства \eqref{ch34:series8} сходится по признаку Лейбница
    $\Rightarrow$ согласно теореме Абеля \eqref{th313}, разложение
    \eqref{ch34:series8} имеет место на отрезке $(-1, 1)$. \\
    В частности, при $x = 1$, получим:
    \begin{gather*}
      1 - \frac{1}{3} + \frac{1}{5} - \frac{1}{7} + \dots = \frac{\pi}{4}
    \end{gather*}
  \item Разложение в ряд $\arcsin x$ \\
    Рассмотрим ряд:
    \begin{gather}
      \frac{1}{\sqrt{1 - t^2}} = 1 + \sum\limits_{n = 1}^{\infty}
      \frac{(2n-1)!!}{(2n)!!}t^{2n}, \ |t| < 1
    \end{gather}
    Интегрирую его почленно по теореме \eqref{th321} от $0$ до $x \in (-1, 1)$
    получим:
    \begin{gather*}
      \int\limits_0^x \frac{dt}{\sqrt{1-t^2}} = \arcsin x = x + \sum\limits_{n
      = 1}^{\infty} \frac{(2n-1)!!}{(2n)!!} \frac{x^{2n+1}}{2n+1}, \ |x| \leq 1
    \end{gather*}
    Справедливость этого разложения при $x = \pm 1$ устанавливается с помощью
    второй теоремы Абеля \eqref{th313}.
\end{itemize}

\section{Формулы Эйлера}
Ряды разложения \eqref{ch34:series1} - \eqref{ch34:series3} функций $e^x, \sin x,
\cos x$ сходятся всюду в комплексной плоскости $\mathbb{C}$. По этой причине
естественны следующие определения($e^z, \sin z, \cos z, z \in \mathbb{C}$):
\begin{gather}
  e^z = \exp(z) = \sum\limits_{n = 0}^{\infty} \frac{z^n}{n!}
  \label{ch35:series1} \\
  \sin z = \sum\limits_{n = 0}^{\infty} \frac{(-1)^n z^{2n+1}}{(2n+1)!}
  \label{ch35:series2} \\
  \cos z = \sum\limits_{n = 0}^{\infty} \frac{(-1)^n z^{2n}}{(2n)!}
  \label{ch35:series3}
\end{gather}
Заменив $z$ сначала на $iz$, а затем на $-iz$ получим:
\begin{gather}\label{ch35:series4}
  e^{iz} = \sum\limits_{n = 0}^{\infty} \frac{i^n z^n}{n!} \\
  e^{-iz} = \sum\limits_{n = 0}^{\infty} \frac{(-1)^n i^n z^n}{n!}
\end{gather}

Заметим, что $i^{2k} = (-1)^k, i^{2k + 1} = (-1)^ki, k = 0, 1, \dots$. \\
\begin{gather*}
  \frac{1}{2} (e^{iz} + e^{-iz}) = \sum\limits_{n = 0}^{\infty} \frac{(-1)^n
  z^{2n}}{(2n)!} \\
  \frac{1}{2i} (e^{iz} - e^{-iz}) = \sum\limits_{n = 0}^{\infty}
  \frac{(-1)^n z^{2n+1}}{(2n+1)!}
\end{gather*}
Сравнив эти формулы с \eqref{ch35:series2}, \eqref{ch35:series3} заключаем, что
\begin{gather}
  \cos z = \frac{e^{iz} + e^{-iz}}{2}
  \label{ch35:eq1} \\
  \sin z = \frac{e^{iz} - e^{-iz}}{2i}
  \label{ch35:eq2}
\end{gather}

Из этих формул следует формула:
\begin{gather}
  \cos z + i \sin z = e^{iz} \label{ch35:eq3}
\end{gather}
Формулы \eqref{ch35:eq1}, \eqref{ch35:eq2} и \eqref{ch35:eq3} называются
формулами Эйлера. \\
Если в формуле \eqref{ch35:eq3} $z = \varphi, \ \varphi \in
\mathbb{R}$, то
\begin{gather*}
  \cos \varphi + i \sin \varphi = e^{i\varphi}
\end{gather*}
Поэтому $z \in \mathbb{C}, |z| = r, \ z = r(\cos \varphi + i \sin
\varphi)$
\begin{gather*}
  z = r e^{i\varphi}
\end{gather*}

\begin{definition}
  $W(x) = U(x) + iV(x), \ x \in \mathbb{R}, V(x) \in \mathbb{R}$ \\
  Положим $\frac{dW}{dx} = U'(x) + iV'(x)$, тогда
  \begin{gather*}
    \int\limits_a^b W(x) dx = \int\limits_a^b U(x) dx + i \int\limits_a^b V(x)
    dx
  \end{gather*}
\end{definition}
$e^{i\pi} = -1$
