\documentclass[12pt]{article}
%\usepackage{ucs}
\usepackage[utf8x]{inputenc} % Включаем поддержку UTF8
\usepackage[russian]{babel}  % Включаем пакет для поддержки русского языка
\usepackage{amsmath}
\usepackage{amssymb}

\title{Дискретная математика}

\date{}
\author{abcdw}

\begin{document}
    \maketitle
    метод включения исключения, формула решета. \newline
    $S,|n|$ \newline
    $N: p_1, p_2, \dots, p_n$
    $n(p_{i_1}, p_{i_2}, p_r)$ - число элементов, которые обладают этими свойствами. \newline
    $n(\overline p) = n - n(p)$ \newline
    $p_1, \dots, p_N$ - совместны. \newline 
    $n(\overline p_1, \dots, p_N) = n - \sum\limits_{i=1}^{N}n(p_i) + \sum n(p_i, p_j) - \dots$ \newline
    Беспорядок - любая такая перестановка, где не один элемент не занимает предназначенное ему место. \newline
    $D_n = n! - c_n^1(n-1)! + c_n^2(n-2)! + \dots + (-1)^nc_n^n(n-n)!$ \newline
    $D_n = n!\left(1 - \frac1{1!} + \frac1{2!} - \dots + (-1)^n\frac1{n!}$ \right) \newline
    $D_n = n! e^{-1}$ \newline
    Следствие. задача Монгор. \newline
    В гардероб сданы шляпы, шляпы перепутаны. Какова вероятность. 
    $D_n = nD_{n-1} + (-1)^n$ \newline
    $D_{n,r}$ - на своих местах могут оставаться r элементов. \newline
    $V(p_1, \dots, p_r)$ -  
    Если во множестве нет ни одного такого элемента. \newline
    Сумма весов элементов, обладающих ровно n свойствами. \newline
    $S, n, N$ найти сумму весов элементов, необладающих ни одним из свойств. \newline   
    $V_{N}(0) = V(0) - V(1) + V(2) + \dots + (-1}^NV(N)$ \newline

\end{document}
