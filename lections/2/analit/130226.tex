\documentclass[12pt]{article}
%\usepackage{ucs}
\usepackage[utf8x]{inputenc} % Включаем поддержку UTF8
\usepackage[russian]{babel}  % Включаем пакет для поддержки русского языка
\usepackage{amsmath}
\usepackage{amssymb}

\title{Аналитическая геометрия и линейная алгебра}

\date{}
\author{abcdw}

\begin{document}
    \maketitle
    $A = (a_{\downarrow}^1, \dots, a_{\downarrow}^n)$ \newline
    $B = (b_{\downarrow}^1, \dots, b_{\downarrow}^n)$ \newline
    $b_{\downarrow}^1 = a_{\downarrow}^1+ \lambda a_{\downarrow}^2, b_{\downarrow}^n = a_{\downarrow}^n$ \newline
    $Rg A \leq Rg B$ \newline
    $C \begin{pmatrix}
        c_{11} & c_{12} & \dots & c_{1n} \\
        0 & c_{21} & \dots & c_{2n} \\  
        \dots & \dots & \dots & \dots \\
        0 & \dots & c_{rr} \dots & c_{rn} \\
        0 & \dots & \dots & 0 \\
        0 & \dots & \dots & 0 
    \end{pmatrix}$ - трапецевидная матрица $m \times n$ \newline
    Теорема. Любая матрица может быть приведена к трапецевидной матрице с помощью элементарных преобразований. \newline
    Доказательство. привести алгоритм. \newline
    ранг трапецевидной матрицы равен числу ненулевых строк. \newline

    Теория систем линейных алгебраических уравнений.
    \S1 Основные понятия. \newline
    СЛАУ - называется система вида: \newline
    $a_{11} x_1 + a_{12} x_2 + \dots + a_{1n} = b_1$ \newline
    $a_{21} x_1 + a_{22} x_2 + \dots + a_{2n} = b_2$ \newline
    $\dots$ \newline
    $a_{m1} x_1 + a_{m2} x_2 + \dots + a_{mn} = b_m$ \newline
    $A = (a_{ij})$ -  матрица системы. \newline
    $x_{\downarrow}$ - столбец неизвестных. \newline
    $b_{\downarrow}$ - столбец свободных членов. \newline
    $\tilde A = (A|b_{\downarrow})$ - расширенная матрица. \newline
    $b_{\downarrow} = 0$ однородная система. \newline
    $Ax_\downarrow = b_\downarrow$ \newline
    Упорядоченный набор из n чисел называется решение, если при его подстановке уравнения обращаются в равенства. \newline
    Если система один имеет хотя бы одно решение, то она называется совместной, иначе несовместной. \newline
    Две СЛАУ с n неизвестными называются эквивалентными, если каждое решение одной системы являеется решением другой и наоборот или если они обе несовместны. \newline
    Общим решением СЛАУ один называется решение, зависящее от нескольких параметров и такое, что всякое частное решение получается из него при помощи специального выбор значений этих параметров.

    \S2 Квадратные СЛАУ и теорема Крамера. \newline
    Теорема. Если определитель матрицы A не равен нулю, то существует и при том единственное решение системы, которое находится или вычисляется по формулам. $x_k = \frac{\triangle_k}{\triangle}$ \newline
    $\triangle_1 = det(b_\downarrow, a_{\downarrow}^2,\dots, a_{\downarrow}^n)$ \newline
    $x_\downarrow = A^{-1} b_\downarrow$ \newline
    $Ax_\downarrow = AA^{-1} b_\downarrow = Eb_\downarrow = b_\downarrow$ - существование. \newline
    $A\tilde x = b, A^{-1}A\tilde x = A^{-1}b \Leftrightarrow \tilde x = A^{-1} b = x$ \newline
    % Доказать формулы Крамера.

    \S3 \newline
    Элементарными преоразованиями СЛАУ будем называть следующие действия: \newline
    1. Перестановка местами двух уравнений. \newline
    2. Умножение на число уравнения. \newline
    3. Прибавление к какому либо  уравнению уравнения умноженного на число. \newline
    Элементарные преобразования СЛАУ приводят к соответствующим преобразованиям расширенной матрицы системы. \newline

    Теорема. Элементарные преобразования СЛАУ переводят ее в эквивалентную. \newline
    Доказательство. Очевидно. \newline
    $Rg A = Rg \tilde A or Rg A + 1 = Rg \tilde A$ \newline

    Метод Гаусса, метод последовательного исключения неизвестных. \newline
    Первый шаг - перестановкой местами уравнений добьемся, чтобы $A_{11} \not = 0$ \newline
    Вторым шагом исключим из уровнения $x_1$ \newline
\end{document}
