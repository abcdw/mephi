\documentclass[12pt]{article}
%\usepackage{ucs}
\usepackage[utf8x]{inputenc} % Включаем поддержку UTF8
\usepackage[russian]{babel}  % Включаем пакет для поддержки русского языка
\usepackage{amsmath}
\usepackage{amssymb}

\title{Аналитическая геометрия и линейная алгебра}

\date{}
\author{abcdw}

\begin{document}
    \maketitle
    $A = (a_{ij}) - m \times n$ матрица. \newline
    Выберем к строк и к столбцов. ($k\leq min(m, n)$). \newline
    $1 \leq i_1 \leq i_2 \leq \dots \leq i_k \leq m; 1 \leq j_1 \leq \dots \leq j_k \leq n$ \newline
    $M_{i_1i_2\dots i_k}^{j_1j_2\dots j_k}$Минором называется определитель к-того порядка($M_k$), составленный из элементов стоящих на пересечении выбранных к строк и столбцов. \newline
    Под минором первого порядка подразумевается элемент. \newline
    $C_n^k C_m^k$ - миноров к-того порядка. \newline

    $A = \begin{pmatrix}
        2 & -3 & -5 & 7 & 4 & 0\\
        1 & -2 & 1 & -1 & 2 & 3\\
        5 & -4 & 8 & 6 & 1 & -7
    \end{pmatrix}$
    $M_{13}^{25} = \begin{vmatrix}
        -3 & 4 \\
        -4 & 1
    \end{vmatrix}
    $

    Все миноры к-ого порядка = 0, тогда у матрицы A все миноры большего порядка = 0, если они существуют. \newline
    Рассмотрим $M_{k+1} = \sum\limits_{i=1}^{k + 1} a'_{1i}A_{1i}^{(k)}, A_{1i}^{(k)} = (-1)^{i+1}M_{1i}^{(k)}$ \newline

    Число $r \in N$ называется рангом матрицы $A = (a_{ij}) - m \times n$, если среди миноров порядка r имеется хотя бы один ненулевой, а все миноры более высокого порядка, если они существуют равны нулю. \newline
    Ранг матрицы - это максимальный порядок минора отличного от нуля. \newline
    $0 \leq Rg A \leq min(m, n)$ \newline
    Пусть матрица $A \not = 0$, тогда r - ее ранг. Любой минор $M_r \not = 0$ называется базисным минором, а строки и столбцы пересечением которых образован этот минор называются базисными строками(столбцами). \newline

    Пример. \newline
    $A = \begin{pmatrix}
        2 & -1 & 0 & 1 & 3 \\
        0 & 2 & 1 &-4 & 1 \\
        4 & -2 & 0 & 2 & 6
    \end{pmatrix}
    \Rightarrow Rg A = 2$, т. к. $M_{12}^{23} = \begin{vmatrix}
        -1 & 0 \\
        2 & 1
    \end{vmatrix} \not = 0, M_3 = 0$ \newline
    Базисные строки и базисные столбцы определяются неоднозначно. \newline

    Теорема о базисном миноре. \newline
    Пусть $A = (a_{ij}) - m \times n$ матрица, $A \not = 0$. \newline
    Базисные строки(столбцы) матрицы A ЛНЗ. \newline
    Каждая строка(столбец) м. A линейно выражается через базисные строки(столбцы) матрицы A. \newline

    Доказательство проведем для столбцов. \newline
    Покажем, что базисные столбцы ЛНЗ. \newline
    По критерию ЛЗ $\Rightarrow$ один базисный столбец линейно выражается через остальные. \newline
    Считаем, что базисный минор имеет вид: \newline
    $M_r = M_{12\dots r}^{1\dots r}$, если это не так, то поменяем строки и столбцы местами. \newline
    Заметим, что при перестановке строк и столбцов ранг очевидно не меняется. \newline
    $M_r = \begin{vmatrix}
        a_{11} & a_{12} & \dots & a_{1r} \\
        a_{21} & a_{22} & \dots & a_{2r} \\
        \vdots & \vdots & \ddots & \vdots \\
        a_{r1} & a_{r2} & \dots & a_{rr}
    \end{vmatrix}, \triangle = \begin{vmatrix}
        a_{11} & a_{12} & \dots & a_{1r} & a_{1j} \\
        a_{21} & a_{22} & \dots & a_{2r} & a_{2j} \\
        \vdots & \vdots & \ddots & \vdots & \vdots \\
        a_{r1} & a_{r2} & \dots & a_{rr} & a_{rj} \\
        a_{i1} & a_{i2} & \dots & a_{ir} & a_{ij}
    \end{vmatrix} = 0$
    Возьмем и зафиксируем j, получим: \newline
    $\triangle = C_1 a_{i1} + C_2a_{i2} + \dots + C_ra_{ir} + M_r a_{ij} = 0, \forall i = \overline{1, m}$
    $a_j = \alpha_1 a_1 + \dots \alpha_r a_r, \alpha_i = -\frac{C_i}{M_r}$ \newline
    Если $\triangle$ раскладывать не по последнему столбцу, а по последней строке, то получим разложение для строк. \newline

    Если $RgA = r$, то любые r+1 строк зависимы. \newline
    $A = \vec a_1, \dots \vec a_2, B = \begin{pmatrix}
        \vec a_1 \\
        \vec a_2 \\
        \vdots \\
        \vec a_{r+1}
    \end{pmatrix}$ \newline
    $Rg B \leq Rg A = r$ \newline
    Поэтому хотя бы одна строка матрциы B выражается через другие строки. \newline
    Отсюда по критерию линейной зависимости строки $\vec a_1 \dots \vec a_{r+1}$ ЛЗ. \newline

    Критерий равенства нулю определеителя. \newline
    Определитель матрицы A $n \times n$ равен нулю, тогда и только тогда когда его строки(столбцы) линейно зависимы. \newline
    $det A = 0 \Rightarrow Rg A \leq n - 1 \Rightarrow$ строки(столбцы) линейно зависимы, так как хотя бы один столбец или строка не являются базисными. \newline
    Пусть строки ЛЗ, тогда по критерию ЛЗ одна из строк линейно выражается, через остальные $\Rightarrow$ det A = 0. \newline

    Теорема о ранге матрицы. \newline
    Максимальное число ЛНЗ строк матрицы A равно макисмальному числу ЛНЗ столбцов и равно $Rg A$ \newline
    Пусть $r \in N = Rg A$, тогда существует r линейно независимых строк(столбцов). \newline
    Любые p > r строк обязательно ЛЗ. \newline


    \S2 Элементарные преобразования матрицы, вычисление ранга методом элементарных преобразований. \newline

    Пусть дана матрица $A = (a_{ij}) - m \times n$ матрица. \newline
    Элементарными преобразованиями матрицы A называются следующие операции, проводимые над строчками(столбцами) матрицы А. \newline
    1. Перестановка. \newline
    2. Умножение строки(столбца) на ненулевое число. \newline
    3. Прибавление к какой-либо строке(столбцу) другой строки(столбца), умноженной на любое число. \newline

    Все три элементарных преобразования обратимы. \newline
    $A \backsim  B$ \newline

    Элементарные преобразования не влияют на ранг матрицы. \newline
    При элементарных преобразованиях 1) и 2) миноры матрицы B отличаются лишь быть может порядком строк, ненулевым множителем. \newline
    Докажем инвариантность ранга для 3) \newline
\end{document}
